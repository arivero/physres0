\documentclass[11pt,a4paper]{article}

\usepackage{amsmath,amssymb,amsfonts}
\usepackage{geometry}
\usepackage{booktabs}
\usepackage{graphicx}
\usepackage{hyperref}

\geometry{margin=2.5cm}
\hypersetup{colorlinks=true,linkcolor=blue,citecolor=blue,urlcolor=blue}

\newcommand{\dd}{\mathrm{d}}
\newcommand{\ee}{\mathrm{e}}
\newcommand{\ii}{\mathrm{i}}
\newcommand{\Tr}{\mathrm{Tr}}
\newcommand{\vev}[1]{\langle #1 \rangle}

\title{Structureless Composites:\\
Form Factor Suppression via Fermionic Exchange}
\author{A.~Rivero \and Claude (Anthropic)}
\date{13 February 2026}

\begin{document}
\maketitle

\begin{abstract}
We study the form factor of a composite particle made of two scalar
(bosonic) constituents bound by fermionic exchange.
Using spectral-function analysis, partial-wave decomposition, and
variational bound-state calculations, we show that fermionic exchange
produces an interaction that is more short-ranged than bosonic exchange
at the same mass scale, leading to a composite that appears
\emph{structureless} at low momentum transfer.

The key mechanism is the \textbf{parity-forced centrifugal barrier}:
the intrinsic parity of a fermion--antifermion pair, $P = (-1)^{L+1}$,
forces the pair into P-wave ($L = 1$) for scalar coupling, adding a
centrifugal barrier that suppresses the spectral function near threshold
as $\delta^{3/2}$ instead of $\delta^{1/2}$.  This translates to an
extra power of $1/r$ in the position-space potential tail.

Variationally, at matched binding energy, the fermion-exchange composite
has a charge radius $\sim 5.8\times$ smaller than a Yukawa (tree-level
boson exchange) composite.  For mediator masses above $\sim 3$~GeV, the
composite is below current experimental limits ($\Lambda > 8$~TeV) on
lepton compositeness.  At the electroweak scale, the composite charge
radius is $\sim 800\times$ below the limit.
\end{abstract}

\tableofcontents

% ====================================================================
\section{Introduction}
\label{sec:intro}
% ====================================================================

Supersymmetric pairing between a composite boson (e.g.\ the pion,
a $\bar{q}q$ bound state) and a fermion (e.g.\ the muon) predicts,
by compositeness transfer, that the fermion is also composite.
Yet the muon is structureless in all scattering experiments to date,
with limits $\Lambda_\mu > 8$~TeV from LEP contact-interaction
analyses, corresponding to
$\sqrt{\vev{r^2}} < 0.025$~fm~\cite{PDG2024}.

This paper resolves the apparent paradox: when binding is mediated by
\emph{fermionic} exchange (the SUSY partner of gluonic binding), the
composite particle has a dramatically suppressed form factor compared
to an equivalent bosonic-exchange composite.  The suppression arises
from a combination of three effects:
\begin{enumerate}
\item \textbf{Selection rule}: Single-fermion exchange between scalar
  sources is forbidden by angular-momentum conservation.  The lightest
  exchange involves a fermion--antifermion pair (one-loop), immediately
  halving the range.
\item \textbf{Parity-forced centrifugal barrier}: The intrinsic parity
  $P = (-1)^{L+1}$ of the fermion pair forces P-wave ($L=1$) for
  natural-parity couplings, adding a centrifugal barrier that
  suppresses the spectral function near threshold.
\item \textbf{Steeper position-space tail}: The spectral suppression
  translates, via Laplace transform, to an extra power of $1/r$ in the
  long-distance potential.
\end{enumerate}

The model we study consists of two scalar bosonic constituents $\phi$
coupled to a Dirac fermion~$\psi$ via a Yukawa interaction
$g\,\phi\,\bar\psi\psi$.  At one loop, the fermion pair generates an
attractive static potential between the $\phi$ sources, which can
bind them into a composite.  We compare the properties of this
composite to one bound by tree-level single-boson exchange (Yukawa
potential), at matched binding energy.

% ====================================================================
\section{The Model}
\label{sec:model}
% ====================================================================

Consider two species of scalar field $\phi_1, \phi_2$ (the bosonic
constituents) and a Dirac fermion $\psi$ (the fermionic mediator).
The relevant interaction Lagrangian is
\begin{equation}
  \mathcal{L}_{\text{int}} = g\,\phi_a\,\bar\psi\psi
  \qquad (a = 1, 2).
\end{equation}
At tree level, no static potential is generated between $\phi_1$ and
$\phi_2$ by single-$\psi$ exchange (the fermion propagator connects
a $\phi\bar\psi\psi$ vertex to another $\phi\bar\psi\psi$ vertex,
but the resulting fermion line has nowhere to close, violating fermion
number --- equivalently, the Dirac propagator is not a scalar under
Lorentz transformations, and cannot produce a static spin-0 potential
from two scalar vertices).

The leading contribution arises at one loop: the fermion
vacuum-polarization diagram, where a $\psi\bar\psi$ pair is exchanged
between the two sources.

\subsection{Static potential from spectral representation}

The static potential is obtained from the K\"all\'en--Lehmann spectral
representation of the vacuum-polarization function $\Pi(q^2)$:
\begin{equation}
  V(r) = -\frac{1}{4\pi r}
  \int_{4m_f^2}^{\infty} \frac{\dd s}{2\pi}\,
  \rho(s)\,\ee^{-\sqrt{s}\,r},
  \label{eq:V-spectral}
\end{equation}
where $\rho(s) = 2\,\mathrm{Im}\,\Pi(s)$ is the spectral function
and $m_f$ is the fermion mass.  The threshold is $s_0 = 4m_f^2$
(pair-production threshold).

For comparison, the tree-level Yukawa potential from exchanging a
single boson of mass $m$ is
\begin{equation}
  V_{\text{Yuk}}(r) = -\frac{g^2}{4\pi}\,\frac{\ee^{-mr}}{r}.
  \label{eq:V-Yukawa}
\end{equation}

% ====================================================================
\section{Spectral Function Analysis}
\label{sec:spectral}
% ====================================================================

\subsection{Fermion loop: scalar coupling}

\begin{proposition}[Fermion spectral function, scalar coupling]
\label{prop:fermion-scalar}
For the coupling $g\,\phi\,\bar\psi\psi$, the spectral function near
threshold $s = 4m_f^2 + \delta$ ($\delta \to 0^+$) behaves as
\begin{equation}
  \rho_F(\delta) \propto \delta^{3/2}.
  \label{eq:rho-fermion}
\end{equation}
\end{proposition}

\begin{proof}
The imaginary part of the one-loop self-energy from the fermion loop is
\begin{equation}
  \mathrm{Im}\,\Pi_F(s)
  = \frac{g^2}{8\pi}\,\sqrt{s}\,\beta^3,
\end{equation}
where $\beta = \sqrt{1 - 4m_f^2/s}$ is the fermion velocity in the
center-of-mass frame.

\textit{Derivation.}
The Dirac trace gives
$\Tr[(\slashed{k}+m_f)(\slashed{k}+\slashed{q}+m_f)]
= 4[k{\cdot}(k{+}q) + m_f^2]$.
After Feynman parameterization, the spectral function in the physical
region $s > 4m_f^2$ is:
\begin{equation}
  \mathrm{Im}\,\Pi_F(s)
  \propto \int_{x_-}^{x_+} \dd x\,
  [m_f^2 - x(1-x)s],
\end{equation}
where $x_\pm = (1 \pm \beta)/2$.
The integrand $m_f^2 - x(1-x)s$ \textbf{vanishes} at threshold
($s = 4m_f^2 \Rightarrow x_- = x_+ = 1/2$, and
$m_f^2 - \tfrac{1}{4} \cdot 4m_f^2 = 0$).

Evaluating the integral (verified symbolically in Appendix~\ref{app:sympy}):
\begin{equation}
  \int_{x_-}^{x_+} [m_f^2 - x(1-x)s]\,\dd x
  = -\frac{\beta\,s\,\beta^2}{6}
  = -\frac{s\,\beta^3}{6}.
\end{equation}
Since $\beta \sim \delta^{1/2}/\sqrt{s_0}$ near threshold, we get
$\mathrm{Im}\,\Pi_F \propto \beta^3 \propto \delta^{3/2}$.
\end{proof}

\subsection{Scalar loop (comparison)}

\begin{proposition}[Scalar spectral function]
\label{prop:scalar}
For a scalar mediator loop ($g_s\,\phi\,|\chi|^2$ with complex scalar
$\chi$ of mass~$m$), the spectral function near threshold is
\begin{equation}
  \rho_S(\delta) \propto \delta^{1/2}.
  \label{eq:rho-scalar}
\end{equation}
\end{proposition}

\begin{proof}
The scalar loop has no Dirac numerator structure.  The imaginary part
from two-body phase space alone gives
$\mathrm{Im}\,\Pi_S(s) \propto \beta \propto \delta^{1/2}$.
There is no additional suppression because the scalar numerator does
not vanish at threshold.
\end{proof}

\subsection{Numerical verification}

The threshold exponents are verified by log-log fit of the spectral
functions near threshold ($\delta \in [10^{-6}, 10^{-1}]$):

\medskip
\begin{center}
\begin{tabular}{lcc}
\toprule
Spectral function & Measured $\alpha$ & Theory \\
\midrule
Fermion, scalar coupling (${}^3P_0$) & 1.4995 & 3/2 \\
Fermion, pseudoscalar coupling (${}^1S_0$) & 0.4992 & 1/2 \\
Scalar loop & 0.4995 & 1/2 \\
\bottomrule
\end{tabular}
\end{center}
\medskip

All three agree with theory to better than 0.1\%.

\begin{figure}[ht]
\centering
\includegraphics[width=0.9\textwidth]{fig_spectral.pdf}
\caption{Left: spectral functions near threshold on a log-log scale.
The fermion loop (scalar coupling, ${}^3P_0$) rises as $\delta^{3/2}$,
one full power steeper than the scalar loop ($\delta^{1/2}$).
Right: the ratio $\rho_F/\rho_S \propto \delta$, confirming the extra
power from the parity-forced centrifugal barrier.}
\label{fig:spectral}
\end{figure}

% ====================================================================
\section{The Partial-Wave Theorem}
\label{sec:partial-wave}
% ====================================================================

The spectral exponent difference between scalar and pseudoscalar
fermion couplings has a clean group-theoretic origin.

\subsection{Quantum numbers of a fermion--antifermion pair}

A $\bar\psi\psi$ pair with relative orbital angular momentum $L$ and
total spin $S$ has quantum numbers
\begin{equation}
  P = (-1)^{L+1}, \qquad
  C = (-1)^{L+S}, \qquad
  J \in \{|L-S|,\ldots,L+S\}.
\end{equation}
The factor $(-1)^{L+\mathbf{1}}$ (not $(-1)^L$) is the intrinsic parity
of the fermion--antifermion system: fermion and antifermion have
\emph{opposite} intrinsic parity (a consequence of the Dirac equation).

\subsection{Threshold behavior}

Near the pair-production threshold
($\beta = \sqrt{1 - 4m_f^2/s} \to 0$), the partial-wave spectral
function behaves as
\begin{equation}
  \rho_L(s) \sim \beta^{2L+1}.
\end{equation}
This is the centrifugal barrier suppression.

\subsection{Theorem: parity-forced centrifugal barrier}

\begin{theorem}
\label{thm:parity}
Let a scalar source ($J^P = 0^+$) couple to a fermion--antifermion pair.
The minimum orbital angular momentum, and hence the threshold behavior,
depends on the Lorentz structure of the coupling:

\medskip
\begin{center}
\begin{tabular}{lcccc}
\toprule
Coupling & $J^{PC}$ & Pair state & $L_{\min}$ & $\rho$ \\
\midrule
$g\phi\bar\psi\psi$ (scalar) & $0^{++}$ & ${}^3P_0$ & 1
  & $\sim\delta^{3/2}$ \\
$g\phi\bar\psi\gamma^5\psi$ (pseudoscalar) & $0^{-+}$ & ${}^1S_0$ & 0
  & $\sim\delta^{1/2}$ \\
\bottomrule
\end{tabular}
\end{center}
\end{theorem}

\begin{proof}
\textit{Case~1 (scalar coupling).}
The pair must have $J^{PC} = 0^{++}$.  From $P = (-1)^{L+1} = +1$,
we need $L$ odd.  The minimum is $L = 1$.  For $J = 0$ with $L = 1$:
$S = 1$ (spin triplet).  The pair state is ${}^3P_0$.
The threshold behavior is $\rho \sim \beta^{2\cdot 1+1} = \beta^3
\sim \delta^{3/2}$.

\textit{Case~2 (pseudoscalar coupling).}
The pair must have $J^{PC} = 0^{-+}$.  From $P = (-1)^{L+1} = -1$,
we need $L$ even.  The minimum is $L = 0$.  For $J = 0$ with $L = 0$:
$S = 0$ (spin singlet).  The pair state is ${}^1S_0$.
The threshold behavior is $\rho \sim \beta^{2\cdot 0+1} = \beta
\sim \delta^{1/2}$.
\end{proof}

\subsection{Corollary: range suppression}

For scalar coupling to a fermion pair, the centrifugal barrier from
$L = 1$ adds one full power of $\delta$ to the spectral function
compared to S-wave.  By the Laplace-transform argument
(Section~\ref{sec:tail}), this translates to one extra power of $1/r$
in the position-space potential tail:

\medskip
\begin{center}
\begin{tabular}{lccc}
\toprule
Coupling & $L$ & $\alpha$ & Tail \\
\midrule
Scalar $\bar\psi\psi$ & 1 & 3/2
  & $\ee^{-2m_f r}/r^{7/2}$ \\
Pseudoscalar $\bar\psi\gamma^5\psi$ & 0 & 1/2
  & $\ee^{-2m_f r}/r^{5/2}$ \\
Scalar pair $|\chi|^2$ & 0 & 1/2
  & $\ee^{-2mr}/r^{5/2}$ \\
\bottomrule
\end{tabular}
\end{center}
\medskip

The extra suppression for scalar coupling is a \emph{direct consequence}
of the intrinsic parity of the fermion--antifermion system.

\subsection{Extension to vector and axial couplings}

The same analysis extends to higher-spin couplings:

\medskip
\begin{center}
\begin{tabular}{lcccc}
\toprule
Coupling & $J^{PC}$ & $L_{\min}$ & State & Threshold \\
\midrule
$\bar\psi\psi$ (scalar) & $0^{++}$ & 1 & ${}^3P_0$ & $\delta^{3/2}$ \\
$\bar\psi\gamma^5\psi$ (pseudo) & $0^{-+}$ & 0 & ${}^1S_0$ & $\delta^{1/2}$ \\
$\bar\psi\gamma^\mu\psi$ (vector) & $1^{--}$ & 0 & ${}^3S_1$ & $\delta^{1/2}$ \\
$\bar\psi\gamma^\mu\gamma^5\psi$ (axial) & $1^{++}$ & 1 & ${}^3P_1$ & $\delta^{3/2}$ \\
\bottomrule
\end{tabular}
\end{center}
\medskip

The pattern: couplings with \textbf{natural parity} ($P = (-1)^J$)
force P-wave or higher; couplings with \textbf{unnatural parity}
($P = (-1)^{J+1}$) allow S-wave.

% ====================================================================
\section{Position-Space Potential}
\label{sec:tail}
% ====================================================================

\begin{proposition}[Long-distance tail]
\label{prop:tail}
If the spectral function near threshold behaves as
$\rho(\delta) \sim \delta^\alpha$, then the position-space potential
at large $r$ is
\begin{equation}
  V(r) \sim -\frac{\ee^{-2m_f r}}{r^{\alpha+2}}.
\end{equation}
\end{proposition}

\begin{proof}
Near threshold, set $s = 4m_f^2 + \delta$ with
$\sqrt{s} \approx 2m_f + \delta/(4m_f)$.
Substituting into \eqref{eq:V-spectral}:
\begin{equation}
  V(r) \sim -\frac{\ee^{-2m_f r}}{4\pi r}
  \int_0^\infty \frac{\dd\delta}{2\pi}\,
  \delta^\alpha\,\ee^{-\delta r/(4m_f)}.
\end{equation}
The Laplace transform gives (verified symbolically,
Appendix~\ref{app:sympy}):
\begin{equation}
  \int_0^\infty \delta^\alpha\,\ee^{-\delta r/(4m_f)}\,\dd\delta
  = \Gamma(\alpha{+}1)\,
  \left(\frac{4m_f}{r}\right)^{\!\alpha+1}.
\end{equation}
Including the $1/(4\pi r)$ kernel:
$V(r) \sim \ee^{-2m_f r}/r^{\alpha+2}$.
\end{proof}

\subsection{Numerical verification}

The position-space tails are verified by computing $V(r)$ from the
spectral integral and fitting the power law of
$|V|\cdot r \cdot \ee^{2m_f r}$ vs.\ $r$:

\medskip
\begin{center}
\begin{tabular}{lcc}
\toprule
Potential & Measured $p$ & Theory ($\alpha{+}2$) \\
\midrule
Fermion, scalar coupling & 3.79 & 7/2 = 3.50 \\
Fermion, pseudoscalar & 2.48 & 5/2 = 2.50 \\
Scalar loop & 2.56 & 5/2 = 2.50 \\
\bottomrule
\end{tabular}
\end{center}
\medskip

The fermion scalar coupling gives a steeper tail than both the
pseudoscalar coupling and the scalar loop, confirming the
parity-forced barrier mechanism.

\begin{figure}[ht]
\centering
\includegraphics[width=0.75\textwidth]{fig_potentials.pdf}
\caption{Binding potentials at matched binding energy $E = -0.05$.
The coupling $\lambda$ is tuned for each potential type to produce
the same ground-state energy.  The fermion-loop potential is more
localized (shorter range, steeper falloff) despite requiring a
smaller coupling $\lambda$.}
\label{fig:potentials}
\end{figure}

% ====================================================================
\section{Bound-State Properties}
\label{sec:bound-state}
% ====================================================================

\subsection{Variational method}

To compare the size of composites bound by different potentials, we use
the variational method with the hydrogen-like trial wave function
\begin{equation}
  u(r) = r\,\ee^{-\alpha r},
  \label{eq:trial}
\end{equation}
where $\alpha$ is the variational parameter.  This gives analytic
results for the observables:

\begin{align}
  T &= \frac{\alpha^2}{2M_{\text{red}}},
  \label{eq:T} \\
  \vev{r^2} &= \frac{3}{\alpha^2},
  \label{eq:r2} \\
  F_1(q^2) &= \frac{1}{(1 + q^2/(4\alpha^2))^2}
  \qquad \text{(dipole form factor)}.
  \label{eq:F1}
\end{align}
The potential expectation value $\vev{V}$ is computed numerically for
each potential type.  The total energy
$E(\alpha) = T + \lambda\vev{V}$
is minimized over $\alpha$, and the coupling strength $\lambda$ is
tuned to match a target binding energy.

\subsection{Results at matched binding energy}

The following table shows the variational results for three potential
types at binding energy $E = -0.05$ (in natural units $m_f = 1$):

\medskip
\begin{center}
\begin{tabular}{lccccc}
\toprule
Potential & $\lambda$ & $\alpha$ & $R_{\text{rms}}$
  & $\vev{r^2}$ & $q_{1\%}$ \\
\midrule
Yukawa (tree boson) & $1.5 \times 10^1$
  & 0.833 & 2.079 & 4.32 & 0.118 \\
Scalar loop & $2.0 \times 10^2$
  & 3.690 & 0.469 & 0.220 & 0.524 \\
Fermion loop (scalar cpg) & $3.8 \times 10^0$
  & 4.829 & 0.359 & 0.129 & 0.686 \\
\bottomrule
\end{tabular}
\end{center}
\medskip

Here $R_{\text{rms}} = \sqrt{\vev{r^2}}$ is in units of $1/m_f$, and
$q_{1\%}$ is the momentum transfer at which $|F_1 - 1| = 1\%$.

\paragraph{Key ratios (fermion / Yukawa):}
\begin{itemize}
\item $R_{\text{rms}}$: $0.17 \Rightarrow$ fermion composite is
  \textbf{5.8$\times$ smaller}
\item $\vev{r^2}$: $0.030 \Rightarrow$ \textbf{34$\times$ suppressed}
\item $q_{1\%}$: $5.8 \Rightarrow$ need \textbf{5.8$\times$ higher
  momentum transfer} to resolve structure
\end{itemize}

\paragraph{Parity barrier effect (fermion / scalar loop):}
$R_{\text{fer}}/R_{\text{scl}} = 0.76$, confirming an additional
$\sim 30\%$ size reduction from the parity-forced centrifugal barrier.

\subsection{Scaling with binding energy}

The size ratios are relatively stable across binding energies:

\medskip
\begin{center}
\begin{tabular}{lcccc}
\toprule
$E$ & $R_{\text{fer}}/R_{\text{Yuk}}$
  & $\vev{r^2}$ ratio
  & $q_{1\%}$ ratio
  & $R_{\text{fer}}/R_{\text{scl}}$ \\
\midrule
$-0.01$ & 0.128 & 0.016 & 7.8 & 0.76 \\
$-0.05$ & 0.173 & 0.030 & 5.8 & 0.76 \\
$-0.10$ & 0.203 & 0.041 & 4.9 & 0.77 \\
$-0.50$ & 0.305 & 0.093 & 3.3 & 0.82 \\
\bottomrule
\end{tabular}
\end{center}
\medskip

The suppression is strongest at weak binding (large composites), where
the long-distance tail dominates the wave function
(Figure~\ref{fig:binding}).

\begin{figure}[ht]
\centering
\includegraphics[width=0.75\textwidth]{fig_formfactor.pdf}
\caption{Form factors $F_1(q^2)$ at matched binding energy $E = -0.05$.
The fermion-loop composite (red, dash-dot) stays close to the
point-particle value $F_1 = 1$ over a much wider $q$~range than the
Yukawa composite (blue, solid).}
\label{fig:formfactor}
\end{figure}

\begin{figure}[ht]
\centering
\includegraphics[width=0.75\textwidth]{fig_binding_scan.pdf}
\caption{Size ratios vs.\ binding energy.  The fermion-loop composite
is always smaller than the Yukawa composite, with the suppression
most pronounced at weak binding where the tail dominates.}
\label{fig:binding}
\end{figure}

% ====================================================================
\section{Physical Implications}
\label{sec:physical}
% ====================================================================

\subsection{Calibration to pion}

To set the mass scale, we identify the Yukawa composite with the pion
($R_{\text{rms}} = r_\pi = 0.659$~fm).  This gives an implied
mediator mass
\begin{equation}
  m_f = \frac{R_{\text{Yuk}} \cdot \hbar c}{r_\pi}
  = \frac{2.079 \times 197.3~\text{MeV\,fm}}{0.659~\text{fm}}
  \approx 623~\text{MeV}.
\end{equation}

The fermion composite then has
\begin{equation}
  r_{\text{fer}} = R_{\text{fer}} \cdot
  \frac{\hbar c}{m_f}
  = 0.359 \times \frac{197.3}{623}~\text{fm}
  \approx 0.114~\text{fm}.
\end{equation}

\subsection{Comparison to experimental limits}

The experimental limit on muon compositeness from LEP
contact-interaction analyses is $\Lambda > 8$~TeV, giving
\begin{equation}
  r_{\mu} < \frac{\hbar c}{\Lambda}
  = \frac{197.3~\text{MeV\,fm}}{8000~\text{MeV}}
  \approx 0.025~\text{fm}.
\end{equation}

At the pion-calibrated scale ($m_f \approx 623$~MeV):
\begin{equation}
  r_{\text{fer}} \approx 0.114~\text{fm}
  \quad > \quad r_{\mu,\text{limit}} \approx 0.025~\text{fm}
  \qquad \text{\textbf{(detectable)}}.
\end{equation}

Thus, at QCD-scale mediator masses, even the fermionic composite
would be visible.  However, the composite size scales as $1/m_f$:
\begin{equation}
  \vev{r^2}(m_f) = \vev{r^2}_{\text{ref}}
  \times \left(\frac{m_{f,\text{ref}}}{m_f}\right)^{\!2}.
\end{equation}

\subsection{Minimum mediator mass for undetectability}

Setting $\vev{r^2}(m_f) = \vev{r^2}_{\text{limit}}$:
\begin{equation}
  m_f^{\text{min}} = m_{f,\text{ref}}
  \times \sqrt{\frac{\vev{r^2}_{\text{ref}}}
  {\vev{r^2}_{\text{limit}}}}
  = 623 \times \sqrt{\frac{0.0129}{6.08 \times 10^{-4}}}
  \approx 2900~\text{MeV} \approx 2.9~\text{GeV}.
  \label{eq:m-min}
\end{equation}

\textbf{For mediator masses above $\sim 3$~GeV, the fermion-exchange
composite is below current experimental limits.}

\subsection{At the electroweak scale}

For a mediator at the electroweak scale ($m_f = 100$~GeV):
\begin{equation}
  \vev{r^2}_{\text{EW}}
  = 1.29 \times 10^{-2}~\text{fm}^2
  \times \left(\frac{623}{10^5}\right)^{\!2}
  \approx 5.0 \times 10^{-7}~\text{fm}^2,
\end{equation}
which is $\sim 800\times$ below the experimental limit.

\begin{figure}[ht]
\centering
\includegraphics[width=0.75\textwidth]{fig_scaling.pdf}
\caption{Composite charge radius vs.\ mediator mass.  The horizontal
line is the experimental muon compositeness limit ($\Lambda > 8$~TeV).
At $m_f \gtrsim 3$~GeV, the fermion-loop composite drops below the
limit.}
\label{fig:scaling}
\end{figure}

\subsection{Form factor at experimental energies}

The momentum transfer at which $F_1$ deviates from~1 by 1\% is:
\begin{align}
  q_{1\%}^{\text{Yuk}} &= 0.118 \times 623~\text{MeV}
  \approx 74~\text{MeV}, \\
  q_{1\%}^{\text{fer}} &= 0.686 \times 623~\text{MeV}
  \approx 427~\text{MeV}.
\end{align}
LEP operated at $q_{\text{max}} \sim 100$~GeV, which is
$234\times$ above the fermion composite's resolution scale at
$m_f = 623$~MeV.  At $m_f > 3$~GeV, the resolution scale exceeds
LEP's reach.

% ====================================================================
\section{Universality: Boson vs.\ Fermion Constituents}
\label{sec:universality}
% ====================================================================

A natural question is whether the form factor suppression depends on
the spin of the \emph{constituents} (the particles being bound), or
only on the spin of the \emph{exchange} particle.  We now show that
the central potential---and hence the form factor at leading
order---is universal: it depends only on the exchange mechanism.

\subsection{Composite boson: two bosons + fermion exchange}

This is the case analyzed in Sections~\ref{sec:spectral}--\ref{sec:bound-state}.
Two scalar sources $\phi_1, \phi_2$ interact via one-loop fermion pair
exchange.  The spectral function is $\rho_F(\delta) \propto \delta^{3/2}$,
and the composite charge radius is $\sim 5.8\times$ smaller than Yukawa.

\subsection{Composite fermion: two fermions + fermion exchange}

Consider instead two fermionic constituents $\psi_1, \psi_2$ (like
quarks) interacting via one-loop fermion exchange.  In a SUSY context,
this corresponds to quarks bound by gluino exchange (there is no direct
quark--quark--gluino vertex in SUSY; the coupling goes through a
squark, requiring at least one loop).

In the static limit, the external fermion propagators reduce to
projectors onto the large components:
\begin{equation}
  \bar{u}(p_1)\,\Gamma\,u(p_1) \to (2m_1)\,\delta_{s_1 s_1'}
  \times (\text{vertex factor}),
\end{equation}
where $\Gamma$ is the vertex structure.  The spinor factors multiply the
overall coupling but \emph{do not modify the spectral function}, which is
a property of the internal loop.

Therefore, the static central potential between fermionic sources has
the \textbf{same spectral function} as between bosonic sources:
\begin{equation}
  \rho^{(\text{fermion sources})}(s) = C_F \times
  \rho^{(\text{boson sources})}(s),
\end{equation}
where $C_F$ is a constant factor from the external spinor contractions.
The threshold behavior, position-space tail, and form factor suppression
ratio are all unchanged.

\subsection{Spin-dependent corrections}

For fermionic sources, the full (non-static) potential includes
spin-dependent terms:
\begin{itemize}
\item \textbf{Spin-spin interaction}: $V_{SS}(r) \propto
  (\vec{\sigma}_1 \cdot \vec{\sigma}_2)\,f(r)$
\item \textbf{Spin-orbit}: $V_{LS}(r) \propto
  (\vec{L} \cdot \vec{S})\,g(r)$
\item \textbf{Tensor}: $V_T(r) \propto S_{12}\,h(r)$
\end{itemize}
These are suppressed by $v^2/c^2$ relative to the central potential
in the non-relativistic limit.  They split energy levels and affect
the fine structure but do not qualitatively change the charge radius
or form factor at the level of precision relevant here.

\subsection{Summary of cases}

\medskip
\begin{center}
\begin{tabular}{lllc}
\toprule
Constituents & Exchange & Composite spin & $R/R_{\text{Yuk}}$ \\
\midrule
Boson + boson & Fermion pair (loop) & 0 & $\sim 0.17$ \\
Fermion + fermion & Fermion pair (loop) & 0 or 1 & $\sim 0.17$ \\
Fermion + boson & Fermion (tree?) & 1/2 & model-dependent \\
\midrule
Boson + boson & Boson (tree) & 0 & 1.00 (reference) \\
Fermion + fermion & Boson (tree) & 0 or 1 & $\sim 1.00$ \\
\bottomrule
\end{tabular}
\end{center}
\medskip

The form factor suppression is determined by the \emph{exchange mechanism}
(fermionic vs.\ bosonic), not by the constituent statistics.  Any
composite bound primarily by fermion-pair exchange will be $\sim 5$--$8\times$
smaller in $R_{\text{rms}}$ (or $\sim 30$--$60\times$ smaller in
$\vev{r^2}$) than a bosonic-exchange composite of the same binding energy.

The one exception is the \textbf{fermion + boson} case with tree-level
single-fermion exchange, which is possible when the vertex structure
allows it.  In this case, the dominant contribution is tree-level Yukawa
(range $\sim 1/m_f$), and the composite is \emph{not} unusually small.

% ====================================================================
\section{Discussion}
\label{sec:discussion}
% ====================================================================

\subsection{The action--angle perspective}

The partial-wave theorem provides the rigorous content behind a
qualitative ``action--angle uncertainty'' argument.

For bosonic exchange, the relevant uncertainty relation is
$\Delta E \cdot \Delta t \gtrsim \hbar$ (or equivalently
$\Delta p \cdot \Delta x \gtrsim \hbar$), where both $E$ and $t$ (or
$p$ and $x$) are unbounded.  The only constraint on the range comes
from the mediator mass: $R \sim 1/m$.

For fermionic exchange, the relevant conjugate pair involves the
angular momentum (action) $J$ and the angle $\varphi$:
$\Delta J \cdot \Delta\varphi \gtrsim \hbar$.
The angle is \emph{compact} ($\varphi \in [0, 2\pi)$), so
$\Delta\varphi$ is bounded.  This means $\Delta J$ cannot be made
arbitrarily small: the fermion pair must carry at least the minimum
orbital angular momentum allowed by parity.

The compactness of the angle variable is the root cause of the
contact-like behavior: the conjugate momentum (action/angular momentum)
is quantized and constrained, preventing the virtual fermion pair from
spreading spatially.

\subsection{Comparison of suppression mechanisms}

\medskip
\begin{center}
\begin{tabular}{lll}
\toprule
Effect & Magnitude & Origin \\
\midrule
Pair threshold & $2\times$ shorter range & Minimum mass $2m_f$ \\
Parity barrier ($L=1$) & $1.3\times$ smaller $R$ & $P = (-1)^{L+1}$ \\
Steeper tail ($r^{-7/2}$ vs $r^{-1}$) & $\sim 3\text{--}6\times$
  smaller $R$ & Spectral suppression \\
\midrule
\textbf{Total} & \textbf{5.8$\times$ smaller $R$}
  & \textbf{(at matched binding energy)} \\
\bottomrule
\end{tabular}
\end{center}
\medskip

The dominant effect is the steeper position-space tail, followed by the
range halving from the pair threshold.  The parity barrier provides an
additional $\sim 30\%$ suppression (comparing fermion scalar coupling
to scalar loop, which share the same exponential range but differ by
one power of $1/r$).

\subsection{Model dependence}

Our calculation uses a simple one-loop model with perturbative coupling.
In a more realistic setting:
\begin{itemize}
\item \textbf{Non-perturbative effects}: Strong coupling could modify
  the spectral function away from threshold.  However, the threshold
  behavior $\rho \sim \beta^{2L+1}$ is protected by kinematics (centrifugal
  barrier) and is non-perturbative.
\item \textbf{Higher loops}: Multi-loop contributions have higher
  thresholds ($6m_f$, $8m_f$, \ldots) and are further suppressed.
\item \textbf{Confinement}: If the fermions are confined (as in SUSY QCD
  with gluinos), the spectrum is discrete rather than continuous, but
  the lowest-lying exchange is still a fermion--antifermion composite
  (gluinoball), which is heavy and short-ranged.
\end{itemize}

% ====================================================================
\section{Conclusion}
\label{sec:conclusion}
% ====================================================================

We have shown that a composite particle bound by fermionic exchange is
significantly more compact than one bound by bosonic exchange at the
same mass scale and binding energy.  The primary mechanism is the
\textbf{parity-forced centrifugal barrier}: the intrinsic parity
$(-1)^{L+1}$ of a fermion--antifermion pair forces P-wave at threshold
for natural-parity couplings, adding a centrifugal barrier that
suppresses the spectral function and steepens the position-space tail.

At matched binding energy, the fermion-exchange composite has:
\begin{itemize}
\item Charge radius $5.8\times$ smaller than Yukawa
\item Mean-square radius $34\times$ suppressed
\item Resolution scale $5.8\times$ higher in momentum transfer
\end{itemize}

For mediator masses above $\sim 3$~GeV, the composite is below current
experimental limits on lepton compositeness ($\Lambda > 8$~TeV).
At the electroweak scale ($m_f = 100$~GeV), the suppression is
$\sim 800\times$ below the limit.

A muon-like particle can therefore be composite (as predicted by SUSY
compositeness transfer from the pion) while appearing completely
structureless in scattering, provided the mediating fermion is
sufficiently heavy.

% ====================================================================
\appendix
\section{Symbolic Verification}
\label{app:sympy}
% ====================================================================

The following analytic results are verified symbolically using SymPy
(script: \texttt{fermion/sympy\_verify.py}):

\begin{enumerate}
\item \textbf{Scalar Feynman integral:}
  $\int_{x_-}^{x_+} [m^2 - x(1{-}x)s]\,\dd x = -s\beta^3/6$.
  (Leading term $\propto \beta^3$.)

\item \textbf{Pseudoscalar Feynman integral:}
  $\int_{x_-}^{x_+} [m^2 + x(1{-}x)s]\,\dd x = \beta(s+8m^2)/6$.
  (Leading term $\propto \beta$.)

\item \textbf{Laplace transform:}
  $\int_0^\infty \delta^\alpha \ee^{-b\delta}\,\dd\delta
  = \Gamma(\alpha{+}1)/b^{\alpha+1}$.

\item \textbf{Threshold expansion:}
  $\beta(4m^2{+}\delta) = \sqrt{\delta}/(2m)
  - \delta^{3/2}/(16m^3) + \cdots$

\item \textbf{Dipole form factor:}
  $F_1(q) = [4\alpha^2/(4\alpha^2{+}q^2)]^2$ for trial
  $u = r\ee^{-\alpha r}$.

\item \textbf{Charge radius:}
  $\vev{r^2} = 3/\alpha^2$ (from both $-6\,F_1'(0)$ and direct
  integration).

\item \textbf{Yukawa expectation:}
  $\vev{V_{\text{Yuk}}} = -\alpha^3/[\pi(2\alpha+\mu)^2]$.
\end{enumerate}

% ====================================================================
\section{Numerical Scripts}
\label{app:scripts}
% ====================================================================

All numerical results are reproduced by:

\medskip
\begin{center}
\begin{tabular}{ll}
\toprule
Script & Content \\
\midrule
\texttt{fermion/fermionic\_composite\_form\_factor\_check.py}
  & Spectral exponents \& tails \\
\texttt{fermion/is\_it\_a\_point.py}
  & Variational bound states \\
\texttt{fermion/sympy\_verify.py}
  & Symbolic verification \\
\bottomrule
\end{tabular}
\end{center}

\medskip
Requirements: Python~3.12, NumPy, SciPy, SymPy.

\begin{thebibliography}{9}
\bibitem{PDG2024}
  R.~L.~Workman \textit{et al.} (Particle Data Group),
  Prog.\ Theor.\ Exp.\ Phys.\ \textbf{2024}, 083C01 (2024).
\end{thebibliography}

\end{document}
