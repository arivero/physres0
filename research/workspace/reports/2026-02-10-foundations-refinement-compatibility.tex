\documentclass[11pt]{article}
\usepackage[a4paper,margin=1in]{geometry}
\usepackage{amsmath,amssymb,amsthm,mathtools}
\usepackage{hyperref}

\newtheorem{theorem}{Theorem}
\newtheorem{proposition}{Proposition}
\newtheorem{remark}{Remark}
\newtheorem{definition}{Definition}

\newcommand{\R}{\mathbb{R}}
\newcommand{\C}{\mathbb{C}}
\newcommand{\eps}{\varepsilon}
\newcommand{\e}{\mathrm{e}}

\title{Refinement Compatibility from Newton to Renormalization:\\
Why Composition Forces Oscillatory Amplitudes}
\author{}
\date{2026-02-10}

\begin{document}
\maketitle

\begin{abstract}
Many ``continuum'' constructions in physics are operational limits of iterative refinement:
polygonal approximations in mechanics, time slicing in quantum mechanics, lattice and cutoff
procedures in field theory. The central theme of this paper is a compatibility requirement:
refinements should compose consistently and yield stable observable assignments.
We argue that this requirement makes two structures hard to avoid. First, if weights compose
multiplicatively under cutting while the underlying functional is additive, then exponential
weighting is forced; the oscillatory weight \(\e^{\tfrac{i}{\hbar}S}\) is the canonical instance.
Second, semigroup composition at short times forces rigid normalization exponents (the
universal \(t^{-d/2}\) scaling in \(d\) dimensions). These ``square-root'' normalization laws are
naturally expressed at the amplitude/half-density level and become visible already in finite
dimensions through stationary phase.

The paper is expository but theorem-facing: we state four representative theorem-grade
packages in explicitly scoped settings (static stationary phase, time-sliced consistency under
explicit gates, Gaussian semigroup normalization, and a screened-Abelian Yukawa kernel
example). We also summarize the current boundary between proved scoped results and open global
gates (continuum existence and reconstruction in interacting local QFT, and first-principles
transfer control in non-Abelian confinement).
\end{abstract}

\section{Introduction}
\label{sec:intro}

The phrase ``take the limit as the refinement parameter goes to zero'' hides a genuine
mathematical issue: the limit need not exist, need not be unique, and may depend on
normalization or subtraction conventions. This is not a pathology of quantum theory alone.
Newton's polygonal method in central-force motion is already a refinement construction:
one first proves an exact finite-step invariant and only then defines the smooth statement
as a controlled refinement limit.

This paper proposes a unifying viewpoint:

\begin{quote}
\emph{A physical law is an assignment of observables that is stable under controlled
refinement and compatible with composition.}
\end{quote}

In this view, quantization and renormalization are not treated as unrelated ``add-ons''. They
are two distinct mechanisms for making refinement limits stable when naive limits fail:
\begin{enumerate}
\item \textbf{Oscillatory amplitudes} stabilize composition rules that localize on variational
extrema in a distributional sense.
\item \textbf{Renormalization group (RG) flow} stabilizes divergent refinement procedures by
explicit regulator dependence plus regulator-independent observables.
\end{enumerate}

To avoid over-claiming, we maintain an explicit scope discipline:
\begin{itemize}
\item We do \emph{not} claim a complete interacting continuum construction in \(d=4\).
\item We do \emph{not} treat geometric half-density language as a substitute for analytic
convergence gates.
\item When a statement depends on a model class or on dimension, we tag that dependency
explicitly.
\end{itemize}

\section{Refinement compatibility and composition}
\label{sec:rcp}

\begin{definition}[Refinement compatibility principle (RCP)]
Fix a family of finite-resolution descriptions indexed by a refinement parameter \(\Lambda\)
(for example, a time slicing step, a lattice spacing, or a cutoff scale), together with
refinement maps \(\rho_{\Lambda\to\Lambda'}\) for \(\Lambda'\succcurlyeq \Lambda\).
An observable assignment \(\mathcal O_\Lambda\mapsto \langle \mathcal O_\Lambda\rangle_\Lambda\)
is \emph{refinement compatible} if:
\begin{enumerate}
\item \textbf{Projective consistency:} \(\langle \mathcal O_\Lambda\rangle_\Lambda
=\langle \rho_{\Lambda\to\Lambda'}\mathcal O_\Lambda\rangle_{\Lambda'}\) whenever both sides are defined.
\item \textbf{Stability:} for each \(\mathcal O\) in a specified observable class, the refinement
limit exists in a specified topology:
\(\langle \mathcal O_\Lambda\rangle_\Lambda \to \langle \mathcal O\rangle\) as \(\Lambda\to\infty\)
(or \(\Lambda\to0\), depending on convention).
\end{enumerate}
\end{definition}

RCP is intentionally minimal: it does not assume an ontological ``continuum''.
It is a compatibility requirement for \emph{how} we define objects by refinement.

\paragraph{Why exponentials appear.}
Suppose a history weight \(W\) composes multiplicatively under cutting while the underlying
functional \(S\) composes additively (as actions do under concatenation):
\[
W(\text{piece}_1\circ \text{piece}_2)=W(\text{piece}_1)\,W(\text{piece}_2),
\qquad
S(\text{piece}_1\circ \text{piece}_2)=S(\text{piece}_1)+S(\text{piece}_2).
\]
Under mild regularity assumptions, this forces \(W\propto \exp(c_0 S)\) with \(c_0\) of
dimension \(1/\text{action}\). Dirac's phase-weighting and Feynman's time slicing are the
best-known instances \cite{Dirac1933,Feynman1948}.

\section{Newton's polygonal refinement (finite invariants first)}
\label{sec:newton}

In Newton's central-force polygonal construction, the key point is methodological:
one proves a \emph{finite-step invariant} (equal swept areas per equal time step under a
central impulse) and then defines the smooth areal-velocity law by a refinement limit.
In modern language, this is a template for how to keep refinement-to-zero statements honest.
See \cite{Newton1687} for the original and any standard mechanics text for the torque/areal
velocity identity.

\section{Static prototype: oscillatory amplitude and localization}
\label{sec:static}

The static prototype is finite-dimensional and theorem-grade under standard hypotheses.

\begin{theorem}[Static localization from oscillatory amplitude (scoped)]
\label{thm:static-born}
Let \(f\in C^\infty(\R)\) and \(O\in C_c^\infty(\R)\). Define the oscillatory amplitude
\[
A_\eps(O):=\eps^{-1/2}\int_{\R}\e^{\tfrac{i}{\eps}f(x)}\,O(x)\,dx,\qquad \eps>0.
\]
Assume \(f'\) has a unique simple zero \(x_0\in \mathrm{supp}(O)\), i.e.\ \(f'(x_0)=0\) and
\(f''(x_0)\neq 0\). Then, as \(\eps\to0^+\),
\[
|A_\eps(O)|^2\longrightarrow 2\pi\,\frac{|O(x_0)|^2}{|f''(x_0)|}
=2\pi\langle \delta(f'),|O|^2\rangle,
\]
up to Fourier-normalization convention.
\end{theorem}

\begin{remark}[Multi-critical interference and averaging]
If several critical points contribute, stationary phase produces a finite sum of oscillatory
phases, and \(|A_\eps(O)|^2\) generally contains nonconvergent interference terms.
In nonresonant settings (pairwise distinct critical values), these cross terms vanish under
explicit coarse graining (e.g.\ a Ces\`aro average in \(t=1/\eps\)).
When critical values coincide, coherent blocks must be grouped explicitly before averaging.
\end{remark}

Theorem~\ref{thm:static-born} can be viewed as a toy witness for a broader pattern:
an \emph{amplitude}-level object is the compositionally natural one, while a squaring step
produces a \emph{density}-level object localized on classical extrema.

\section{Dynamics prototype: time slicing and scoped consistency}
\label{sec:dynamics}

For actions on time histories, the variational localization target is formally
\(\delta(\delta S/\delta q)\). Time slicing turns this into an exact finite-dimensional
object \(\delta(\nabla S_N)\) at fixed resolution. The remaining work is to control the
refinement limit.

\begin{theorem}[Scoped dynamic consistency (gate-form statement)]
\label{thm:dynamic-gates}
Consider normalized finite-\(N\) states of the schematic form
\[
\omega_{c,N}(F):=\frac{\int_{\R^N}\e^{-cS_N(x)}F(x)\,dx}{\int_{\R^N}\e^{-cS_N(x)}\,dx},
\qquad c\in\C,\ \Re c>0,
\]
on an explicit cylinder observable class.
Assume the following gates on compact \(c\)-domains:
\begin{enumerate}
\item projective compatibility under refinement \(N\mapsto N'\),
\item uniform non-vanishing of denominators,
\item uniform Cauchy control for cylinder expectations,
\item finite-\(N\) Schwinger--Dyson identities,
\item \(c\)-invariance under a \(c\)-preserving reparameterization flow,
\item controlled de-regularization (a one-sided \(\eta\to0^+\) limit in an admissible class)
commuting with the refinement limit.
\end{enumerate}
Then \(\omega_{c,N}(F)\to\omega_c(F)\) on cylinders, Schwinger--Dyson identities pass to the
limit in the same scope, and the refinement limit of the normalized time-sliced transition
amplitude agrees with \(\omega_c\) on the same observable class.
\end{theorem}

Theorem~\ref{thm:dynamic-gates} is deliberately formulated as a gate checklist: it is a
precise \emph{proof target}. The foundational point is that time slicing does not by itself
provide a continuum object; it provides a finite-resolution object to which RCP can be
applied.

\section{Newton-limit paradox support: semigroup normalization forces \(t^{-d/2}\)}
\label{sec:td2}

One concrete place where refinement compatibility becomes rigid is short-time kernel
composition. In Euclidean signature, the heat kernel is the standard example; here we
highlight the abstract mechanism:
\emph{semigroup composition fixes the normalization exponent.}

\begin{theorem}[Gaussian semigroup normalization exponent (Euclidean, kernel-level)]
\label{thm:td2}
Let \(d\in\mathbb{N}\). Consider translation-invariant kernels \(K_t:\R^d\to\R\) of the Gaussian form
\[
K_t(x)=a(t)\,\exp\!\left(-\frac{|x|^2}{4\,b(t)}\right),
\qquad t>0,
\]
with \(a(t)>0\), \(b(t)>0\), and assume \(b\) is continuous on \((0,\infty)\). Assume further that:
\begin{enumerate}
\item \textbf{Semigroup (convolution) composition:}
\(\displaystyle K_{t+s}=K_t\ast K_s\) for all \(t,s>0\).
\item \textbf{Mass normalization:} \(\displaystyle \int_{\R^d}K_t(x)\,dx=1\) for all \(t>0\).
\end{enumerate}
Then \(b(t)\) is additive (\(b(t+s)=b(t)+b(s)\)), hence \(b(t)=\sigma t\) for some \(\sigma>0\), and
\[
a(t)=\big(4\pi b(t)\big)^{-d/2}\propto t^{-d/2}.
\]
In particular, the diagonal normalization exhibits the universal exponent:
\(\displaystyle K_t(0)\propto t^{-d/2}\).
\end{theorem}

\begin{proof}[Proof sketch]
Semigroup composition of Gaussians forces additivity of the variance parameter \(b(t)\)
(already visible in one dimension by explicit convolution). With \(b(t)=\sigma t\), mass
normalization fixes \(a(t)\) by a direct Gaussian integral:
\(\int_{\R^d}\exp(-|x|^2/(4b))dx=(4\pi b)^{d/2}\).
\end{proof}

Theorem~\ref{thm:td2} is a minimal formal witness for the ``square-root Jacobian'' theme:
composition fixes a half-power exponent (\(d/2\)) that is naturally attached to an amplitude
kernel rather than to a probability density. In semiclassical settings, the same rigidity
reappears through mixed-Hessian determinant prefactors (Van Vleck type formulas), where
eliminating intermediate variables produces Schur-complement determinants.

\section{Renormalization as limit control}
\label{sec:rg}

In interacting field models, refinement limits may diverge.
Renormalization is the disciplined version of the following template:
\begin{quote}
regulate \(\to\) subtract divergence \(\to\) fix finite ambiguity by a normalization condition
\(\to\) take the limit.
\end{quote}
The RG expresses the requirement that changing the intermediate scale does not change
physical observables; in operational terms, it is a compatibility condition for a family of
regulated descriptions \cite{WilsonKogut1974}.

The key foundational separation is:
\begin{itemize}
\item \textbf{Kinematic coherence:} composition rules and coordinate invariances at fixed
regulator (often expressible in geometric kernel language).
\item \textbf{Dynamical convergence:} existence of regulator-removed limits in an explicit
topology, plus persistence of identities (e.g.\ Schwinger--Dyson) under the limit.
\end{itemize}

\section{Example: screened Abelian long-range behavior (dimension-explicit)}
\label{sec:yukawa}

To illustrate the ``tag dependencies explicitly'' rule, consider a screened Abelian
(\(U(1)\)) sector with a mass gap \(m>0\) in spatial dimension \(n=d\) (spacetime
dimension \(D=n+1\)).

\begin{theorem}[Yukawa kernel and asymptotic (screened Abelian, all \(D\))]
\label{thm:yukawa}
Let \(n\ge 1\) and \(m>0\). The unique decaying fundamental solution of
\((-\Delta+m^2)G_{n,m}=\delta_0\) on \(\R^n\) satisfies
\[
G_{n,m}(r)=\frac{1}{(2\pi)^{n/2}}\left(\frac{m}{r}\right)^\nu K_\nu(mr),
\qquad \nu=\frac n2-1,
\]
and as \(r\to\infty\),
\[
G_{n,m}(r)=C_{n,m}\,r^{-(n-1)/2}\e^{-mr}\big(1+O(r^{-1})\big),
\]
hence the inter-source interaction contribution decays exponentially.
\end{theorem}

\begin{proof}[Proof sketch]
Use the Fourier representation \(\widehat{G}(k)=(|k|^2+m^2)^{-1}\), reduce radially, and
apply the standard Hankel/Bessel identity to obtain the modified Bessel \(K_\nu\) form; then
use \(K_\nu(z)\sim \sqrt{\pi/(2z)}\,\e^{-z}\) as \(z\to+\infty\).
\end{proof}

\section{Status and open problems (date-anchored)}
\label{sec:status}

We close with a compact program scorecard as of 2026-02-10.
Scores are internal maturity markers (0--10) used to prioritize proof work:
\begin{itemize}
\item \(10\): theorem-grade closure in intended global scope,
\item \(7\)–\(9\): theorem-grade closure in a strong scoped model with explicit remaining gaps,
\item \(0\)–\(6\): substantial structure but missing key closures.
\end{itemize}

\begin{center}
\small
\renewcommand{\arraystretch}{1.15}
\begin{tabular}{|c|c|p{0.70\linewidth}|}
\hline
Track & Score & Closure boundary (scoped) and open gap \\
\hline
Static amplitude localization & 9 & Theorem~\ref{thm:static-born} is theorem-grade under nondegeneracy; multi-critical interference requires explicit averaging prescriptions. \\
\hline
Time-slicing consistency & 8 & Gate-form closure (Theorem~\ref{thm:dynamic-gates}); model-dependent discharge of denominator/tail/de-regularization gates remains the real frontier. \\
\hline
Kernel normalization lane & 8 & Semigroup normalization forces \(t^{-d/2}\) in Gaussian settings (Theorem~\ref{thm:td2}); extension to oscillatory kernels requires explicit regularization/analytic-continuation gates. \\
\hline
Gauge long-range taxonomy & 8 & Screened Abelian branch is theorem-closed (Theorem~\ref{thm:yukawa}); non-Abelian confinement beyond strong-coupling windows requires first-principles transfer control and dynamical-matter string-breaking theorems. \\
\hline
\end{tabular}
\end{center}

\paragraph{Open problems (selected).}
\begin{enumerate}
\item Close interacting local field-theory continuum existence and reconstruction gates in \(d=4\) in a
theorem-grade framework.
\item Replace scoped transfer assumptions in non-Abelian confinement lanes by first-principles bounds and
treat dynamical-matter crossover rigorously.
\item Make the relationship between amplitude-level half-density calculus and dynamical convergence gates
fully explicit in representative models (beyond Gaussian witnesses).
\end{enumerate}

\begin{thebibliography}{99}

\bibitem{Newton1687}
I. Newton, \emph{Philosophi\ae{} Naturalis Principia Mathematica} (1687).

\bibitem{Dirac1933}
P. A. M. Dirac, ``The Lagrangian in Quantum Mechanics,'' \emph{Physikalische Zeitschrift der Sowjetunion}
\textbf{3} (1933), 64--72.

\bibitem{Feynman1948}
R. P. Feynman, ``Space-Time Approach to Non-Relativistic Quantum Mechanics,''
\emph{Reviews of Modern Physics} \textbf{20} (1948), 367--387.

\bibitem{WilsonKogut1974}
K. G. Wilson and J. Kogut, ``The Renormalization Group and the \(\eps\) Expansion,''
\emph{Physics Reports} \textbf{12} (1974), 75--200.

\bibitem{Horman}
L. H\"ormander, \emph{The Analysis of Linear Partial Differential Operators I} (Springer, 1983).

\bibitem{WatsonBessel}
G. N. Watson, \emph{A Treatise on the Theory of Bessel Functions} (Cambridge University Press, 1922).

\end{thebibliography}

\end{document}
