\documentclass[11pt]{article}
\usepackage[a4paper,margin=1in]{geometry}
\usepackage{amsmath,amssymb,amsthm,mathtools}
\usepackage{bm}
\usepackage{enumitem}
\usepackage{hyperref}

\newtheorem{theorem}{Theorem}[section]
\newtheorem{lemma}[theorem]{Lemma}
\newtheorem{proposition}[theorem]{Proposition}
\newtheorem{corollary}[theorem]{Corollary}
\newtheorem{definition}[theorem]{Definition}
\newtheorem{remark}[theorem]{Remark}

\newcommand{\R}{\mathbb{R}}
\newcommand{\C}{\mathbb{C}}
\newcommand{\eps}{\varepsilon}
\newcommand{\Cyl}{\mathrm{Cyl}}
\newcommand{\Cov}{\mathrm{Cov}}
\newcommand{\e}{\mathrm{e}}

\title{Claim 1 (Scoped): Complete Proof in a Projective Oscillatory Class}
\author{}
\date{2026-02-09}

\begin{document}
\maketitle

\begin{abstract}
We provide a complete proof of Claim 1 in a scoped but nontrivial class:
projective cylinder states built from oscillatory actions with a finite interacting block (allowing mode coupling) and a quartic-stabilized tail.
The proof closes:
(i) exact large-$N$ projective stability,
(ii) continuum state existence on cylinder observables,
(iii) constructive counterterm repair for coefficient drift,
(iv) de-regularization $\eta\to0^+$ via contour rotation (Gaussian, factorized quartic, and coupled quartic block),
and (v) explicit Gaussian semiclassical channel expansion in point-supported distribution modes.
It further closes:
(vi) genuinely large-$N$ mode-coupled lifts, including an explicit Gaussian-tail rate and a non-factorized quartic-tail class under log-derivative summability,
(vii) explicit non-vanishing lower bounds for normalized oscillatory partition factors,
(viii) extension of observables from Gaussian-exponential test families to Schwartz and weighted Sobolev classes,
and (ix) finite-dimensional Schwinger-Dyson identities and exact $\tau_\mu$-type scale-flow covariance.
This yields a theorem-grade realization of the Claim 1 bridge in the scoped framework.
\end{abstract}

\section{Statement of the Scoped Claim}

Background references used in this note include stationary phase and distribution theory
\cite{hormander1,gelfandshilov1}, oscillatory/Feynman integrals \cite{feynman1948,albeverio2008},
and groupoid quantization/tangent-groupoid context \cite{connes1994,landsman1999,landsman1998,landsmanramazan2001}.

\begin{definition}[Projective cylinder system]
For $N\ge1$, let $X_N=\R^N$ and $\pi_{N\to m}:X_N\to X_m$ be coordinate projection ($N\ge m$).
Define
\[
\Cyl:=\bigcup_{m\ge1}\{F=F_m\circ \pi_{\infty\to m}: F_m\in C_b^2(\R^m)\}.
\]
\end{definition}

\begin{definition}[Block-tail action class]
Fix $b\in\mathbb{N}_0$, $g\ge0$, and parameters
\[
0<\lambda_-\le\lambda_j\le\lambda_+,\qquad \kappa_j\in[0,\kappa_+].
\]
For $N\ge b$, define
\[
S_N(x)=P_b(x_1,\dots,x_b)+\sum_{j=b+1}^N q_j(x_j),
\quad
q_j(u)=\frac{\lambda_j}{2}u^2+g\kappa_j u^4.
\]
Assume:
\begin{enumerate}[label=(A\arabic*),leftmargin=1.6em]
\item $P_b$ is a real polynomial with $P_b(0)=0$, $\nabla P_b(0)=0$.
\item There exist $c_4>0,c_2\ge0,C_0\ge0$ such that
\[
P_b(z)\ge c_4\|z\|^4-c_2\|z\|^2-C_0,\qquad z\in\R^b.
\]
\end{enumerate}
\end{definition}

For $\eta>0$ and $\eps>0$, define the normalized oscillatory state
\[
\omega_{\eps,\eta,N}(F_m)
:=
\frac{\int_{\R^N}\e^{-(\eta-i/\eps)S_N(x)}F_m(x_1,\dots,x_m)\,dx}
{\int_{\R^N}\e^{-(\eta-i/\eps)S_N(x)}\,dx},
\qquad N\ge m,
\]
whenever the denominator is nonzero.

\begin{theorem}[Scoped Claim 1, complete proof]
\label{thm:main}
In the block-tail action class:
\begin{enumerate}[label=(\roman*),leftmargin=1.6em]
\item \textbf{Exact projective stability}: for every cylinder observable $F_m$ and $N\ge M:=\max\{m,b\}$,
\[
\omega_{\eps,\eta,N}(F_m)=\omega_{\eps,\eta,M}(F_m).
\]
\item \textbf{Continuum state}: for each $(\eps,\eta)$, there is a unique functional
$\omega_{\eps,\eta}:\Cyl\to\C$ with
\[
\omega_{\eps,\eta}(F_m\circ\pi_{\infty\to m})
 :=
\omega_{\eps,\eta,M}(F_m),\quad M=\max\{m,b\},
\]
and
\[
|\omega_{\eps,\eta}(F)|\le C_{\eps,\eta,m}\,\|F\|_\infty,
\]
where, for $M=\max\{m,b\}$,
\[
C_{\eps,\eta,m}:=
\frac{\int_{\R^M}\e^{-\eta S_M(u)}\,du}
{\left|\int_{\R^M}\e^{-(\eta-i/\eps)S_M(u)}\,du\right|}
<\infty.
\]
\item \textbf{Counterterm repair}: explicit local quadratic/quartic counterterms can repair scale-dependent coefficient drift and restore exact projective stability.
\item \textbf{De-regularization}: for Gaussian-exponential cylinder observables
$F_m(x)=p(x)\e^{-x^\top Bx}$ (polynomial $p$, $B\succeq0$), the limit
\[
\omega_{\eps,0}(F):=\lim_{\eta\to0^+}\omega_{\eps,\eta}(F)
\]
exists (branch fixed by contour angle $\pi/8$).
\item \textbf{Semiclassical channels (Gaussian subcase)}:
if $g=0$, $b=0$, then for $F_m\in\mathcal{S}(\R^m)$,
\[
\omega_{\eps,0}(F_m)=\left[\exp\!\left(\frac{i\eps}{2}\mathcal{L}_m\right)F_m\right]_{x=0},
\quad
\mathcal{L}_m:=\sum_{j=1}^m\lambda_j^{-1}\partial_{x_j}^2,
\]
hence
\[
\omega_{\eps,0}(F_m)
=
\sum_{k=0}^{K-1}\frac{1}{k!}\left(\frac{i\eps}{2}\right)^k(\mathcal{L}_m^kF_m)(0)
+R_{K,\eps}(F_m),
\]
which is precisely a hierarchy of point-supported derivative channels at the extremum.
\end{enumerate}
\end{theorem}

Sections \ref{sec:projective}--\ref{sec:channels} prove each item.

\section{Projective Stability and Continuum State}
\label{sec:projective}

\begin{lemma}[Tail factorization]
\label{lem:factorization}
Let $M=\max\{m,b\}$ and $N\ge M$.
Write $x=(u,v)$ with $u\in\R^M$, $v\in\R^{N-M}$.
Then
\[
S_N(u,v)=S_M(u)+\sum_{j=M+1}^N q_j(v_j).
\]
\end{lemma}

\begin{proof}
By construction, coordinates $1,\dots,b$ appear only in $P_b$, and each $j>b$ contributes only $q_j(x_j)$.
For $N\ge M$, all interacting coordinates are contained in the $u$-block.
\end{proof}

\begin{proposition}[Exact large-$N$ stability]
\label{prop:exact-stability}
Assume denominators are nonzero. Then
\[
\omega_{\eps,\eta,N}(F_m)=\omega_{\eps,\eta,M}(F_m),\qquad N\ge M.
\]
\end{proposition}

\begin{proof}
Using Lemma \ref{lem:factorization},
\begin{align*}
&\int_{\R^N}\e^{-(\eta-i/\eps)S_N(x)}F_m(x_1,\dots,x_m)\,dx\\
&=
\left[\int_{\R^M}\e^{-(\eta-i/\eps)S_M(u)}F_m(u_1,\dots,u_m)\,du\right]
\prod_{j=M+1}^N
\left[\int_{\R}\e^{-(\eta-i/\eps)q_j(t)}\,dt\right].
\end{align*}
The denominator factorizes with the same tail product, which cancels in the ratio.
\end{proof}

\begin{proposition}[Continuum functional on cylinders]
\label{prop:continuum-functional}
For fixed $(\eps,\eta)$, define
\[
\omega_{\eps,\eta}(F_m\circ\pi_{\infty\to m})
:=
\omega_{\eps,\eta,M}(F_m),\quad M=\max\{m,b\}.
\]
This is well-defined, linear on $\Cyl$, and bounded by
\[
|\omega_{\eps,\eta}(F_m\circ\pi_{\infty\to m})|
\le C_{\eps,\eta,m}\|F_m\|_\infty,
\]
with $C_{\eps,\eta,m}$ as in Theorem \ref{thm:main}.
\end{proposition}

\begin{proof}
Well-definedness follows from Proposition \ref{prop:exact-stability}.
Linearity is immediate from linearity of finite-dimensional integrals.
For the bound, write
\[
Z_M:=\int_{\R^M}\e^{-(\eta-i/\eps)S_M(u)}\,du,\qquad
A_M:=\int_{\R^M}\e^{-\eta S_M(u)}\,du.
\]
Then
\[
\left|
\int_{\R^M}\e^{-(\eta-i/\eps)S_M(u)}F_m(u)\,du
\right|
\le
\|F_m\|_\infty A_M,
\]
and therefore
\[
|\omega_{\eps,\eta}(F_m\circ\pi_{\infty\to m})|
\le
\frac{A_M}{|Z_M|}\|F_m\|_\infty
=C_{\eps,\eta,m}\|F_m\|_\infty.
\]
The constant is finite whenever $Z_M\neq0$.
\end{proof}

\section{Counterterm Repair}
\label{sec:counterterms}

Suppose bare coefficients drift with $N$:
\[
\lambda_{j,N}^{\mathrm{bare}}=\lambda_j+r_{j,N},\qquad
\kappa_{j,N}^{\mathrm{bare}}=\kappa_j+s_{j,N}.
\]
Assume bounds
\[
|r_{j,N}|\le \lambda_-/2,\qquad
|s_{j,N}|\le \kappa_+/2.
\]
Define local counterterms
\[
\delta S_N(x)=\sum_{j=1}^N\left[-\frac{r_{j,N}}{2}x_j^2-g\,s_{j,N}x_j^4\right].
\]
Then
\[
S_N^{\mathrm{ren}}:=S_N^{\mathrm{bare}}+\delta S_N
\]
has coefficients exactly $(\lambda_j,\kappa_j)$ and belongs to the stable block-tail class.

\begin{proposition}[Constructive repair]
The renormalized family $S_N^{\mathrm{ren}}$ satisfies the hypotheses of Proposition \ref{prop:exact-stability}; therefore projective stability is restored exactly.
\end{proposition}

\begin{proof}
Direct substitution cancels all coefficient drifts coordinatewise.
The restored action has $N$-independent coefficients and the same block-tail decomposition.
Apply Proposition \ref{prop:exact-stability}.
\end{proof}

\section{De-Regularization \texorpdfstring{$\eta\to0^+$}{eta->0+}}
\label{sec:dereg}

\begin{lemma}[Rotated contour dominance]
\label{lem:dominance}
Fix finite dimension $d$ and polynomial action
\[
\mathcal{S}(x)=Q_2(x)+gQ_4(x),
\]
where $Q_2$ is real quadratic and $Q_4$ is real quartic with
\[
Q_4(y)\ge c\|y\|^4,\qquad c>0.
\]
Let $x=\e^{i\pi/8}y$ and $\eta\in[0,\eta_0]$.
For $F(y)=p(y)\e^{-y^\top B y}$ with polynomial $p$ and $B\succeq0$,
there exist constants $C,\tilde c_4>0,\tilde c_2\ge0$ such that
\[
\left|
\e^{-(\eta-i/\eps)\mathcal{S}(\e^{i\pi/8}y)}
F(\e^{i\pi/8}y)
\right|
\le
C(1+\|y\|^k)\e^{-\tilde c_4\|y\|^4+\tilde c_2\|y\|^2}.
\]
\end{lemma}

\begin{proof}
Under $x=\e^{i\pi/8}y$, quartic monomials acquire phase $\e^{i\pi/2}=i$.
Hence
\[
\Re\!\left(\frac{i}{\eps}gQ_4(\e^{i\pi/8}y)\right)
=
-\frac{g}{\eps}Q_4(y)
\le
-\frac{gc}{\eps}\|y\|^4.
\]
The remaining quadratic and $\eta$-terms contribute at most $+\tilde c_2\|y\|^2$.
Polynomial prefactors produce $(1+\|y\|^k)$.
The right side is integrable on $\R^d$.
\end{proof}

\begin{proposition}[Finite-dimensional \texorpdfstring{$\eta\to0^+$}{eta->0+} limit]
\label{prop:eta-limit-fd}
In the setting of Lemma \ref{lem:dominance}, define
\[
I_\eta(F):=\int_{\R^d}\e^{-(\eta-i/\eps)\mathcal{S}(x)}F(x)\,dx,
\]
with contour branch fixed by angle $\pi/8$.
Then
\[
\lim_{\eta\to0^+}I_\eta(F)=I_0(F).
\]
If $I_\eta(1)\neq0$ for small $\eta$ and $I_0(1)\neq0$, then
\[
\lim_{\eta\to0^+}\frac{I_\eta(F)}{I_\eta(1)}
=
\frac{I_0(F)}{I_0(1)}.
\]
\end{proposition}

\begin{proof}
For $\eta>0$, deform real contour to angle $\pi/8$ (entire integrand, quartic decay on connecting arcs).
On that contour, pointwise convergence as $\eta\to0^+$ is immediate.
Lemma \ref{lem:dominance} gives a common $L^1$ dominator.
Apply dominated convergence to numerator and denominator.
\end{proof}

\begin{corollary}[De-regularized cylinder state]
\label{cor:dereg-cylinder}
For Gaussian-exponential cylinder observables in Theorem \ref{thm:main}, the limit
\[
\omega_{\eps,0}(F)=\lim_{\eta\to0^+}\omega_{\eps,\eta}(F)
\]
exists and is independent of $N$.
\end{corollary}

\begin{proof}
Reduce to stabilized finite dimension $M=\max\{m,b\}$ by Proposition \ref{prop:exact-stability}.
Then apply Proposition \ref{prop:eta-limit-fd} in dimension $M$.
\end{proof}

\section{Gaussian Channel Expansion}
\label{sec:channels}

Now take the Gaussian subcase $g=0$, $b=0$:
\[
S_m(x)=\frac12\sum_{j=1}^m\lambda_jx_j^2.
\]
Define, for $F\in\mathcal{S}(\R^m)$,
\[
\omega_{\eps,0}(F)
:=
\frac{\int_{\R^m}\e^{\frac{i}{\eps}S_m(x)}F(x)\,dx}
{\int_{\R^m}\e^{\frac{i}{\eps}S_m(x)}\,dx}.
\]

\begin{proposition}[Exact operator form]
\label{prop:operator-form}
Let
\[
\mathcal{L}_m=\sum_{j=1}^m\lambda_j^{-1}\partial_{x_j}^2.
\]
Then
\[
\omega_{\eps,0}(F)
=
\left[\exp\!\left(\frac{i\eps}{2}\mathcal{L}_m\right)F\right]_{x=0}.
\]
\end{proposition}

\begin{proof}
Write
\[
F(x)=\frac{1}{(2\pi)^m}\int_{\R^m}\hat F(\xi)\e^{i\xi\cdot x}\,d\xi.
\]
By Gaussian completion (Fresnel branch),
\[
\frac{\int \e^{\frac{i}{2\eps}\sum_j\lambda_j x_j^2}\e^{i\xi\cdot x}\,dx}
{\int \e^{\frac{i}{2\eps}\sum_j\lambda_j x_j^2}\,dx}
=
\exp\!\left(-\frac{i\eps}{2}\sum_{j=1}^m\frac{\xi_j^2}{\lambda_j}\right).
\]
Therefore
\[
\omega_{\eps,0}(F)
=
\frac{1}{(2\pi)^m}\int \hat F(\xi)
\exp\!\left(-\frac{i\eps}{2}\sum_j\frac{\xi_j^2}{\lambda_j}\right)d\xi.
\]
The multiplier is exactly that of $\exp((i\eps/2)\mathcal{L}_m)$ evaluated at $x=0$.
\end{proof}

\begin{corollary}[Point-supported channel hierarchy]
For $K\ge1$,
\[
\omega_{\eps,0}(F)
=
\sum_{k=0}^{K-1}\frac{1}{k!}\left(\frac{i\eps}{2}\right)^k(\mathcal{L}_m^kF)(0)
+R_{K,\eps}(F),
\]
with $R_{K,\eps}(F)=O(\eps^K)$ as $\eps\to0^+$.
Thus channels are derivatives of $F$ at the extremum $x=0$, i.e.\ point-supported distribution modes.
\end{corollary}

\begin{proof}
Expand the exponential operator in power series and use Schwartz regularity.
\end{proof}

\section{Static Extremum Localization and the Variational-Delta Ladder}

\begin{proposition}[Static Morse localization]
Let $f\in C^\infty(\R^d)$ with unique nondegenerate critical point $x_\star$:
\[
\nabla f(x_\star)=0,\qquad \det \nabla^2 f(x_\star)\neq0.
\]
For $O\in C_c^\infty(\R^d)$,
\[
A_\eps(O):=\eps^{-d/2}\int_{\R^d}\e^{\frac{i}{\eps}f(x)}O(x)\,dx
\]
satisfies
\[
|A_\eps(O)|^2
\to
(2\pi)^d\frac{|O(x_\star)|^2}{|\det\nabla^2 f(x_\star)|}.
\]
Equivalently,
\[
|A_\eps(O)|^2\to (2\pi)^d\langle \delta(\nabla f),|O|^2\rangle.
\]
\end{proposition}

\begin{proof}
Standard stationary phase at a single Morse critical point.
\end{proof}

\begin{corollary}[Finite-dimensional QM/QFT truncations]
For any finite-dimensional discretization $S_N$ of a variational action
(time-sliced QM or lattice QFT), the same stationary-phase mechanism localizes on
\[
\nabla S_N=0,
\]
providing the finite-dimensional realization of
$\delta(\delta S)$ as an extremum selector.
\end{corollary}

\section{Large-\texorpdfstring{$N$}{N} Mode-Coupled Lift}

We now pass from fixed interacting blocks to a genuinely growing mode-coupled family.

\begin{theorem}[Large-$N$ coupled Gaussian-tail convergence with rate]
\label{thm:largen-coupled}
Fix $m\ge1$. Let
\[
S_N(u,v)=P_m(u)+\sum_{j=m+1}^N\left(\frac{\lambda_j}{2}+\beta_j(u)\right)v_j^2,
\quad
\beta_j(u)=\sum_{i=1}^m a_{ij}u_i^2,
\]
with:
\begin{enumerate}[label=(B\arabic*),leftmargin=1.6em]
\item $\lambda_j\ge \lambda_->0$,
\item $a_{ij}\ge0$ and $A_j:=\sum_{i=1}^m a_{ij}$ satisfies
\[
\sum_{j=m+1}^\infty \frac{A_j}{\lambda_j}<\infty,
\]
\item $P_m(u)\ge c_4\|u\|^4-c_2\|u\|^2-C_0$.
\end{enumerate}
For bounded $F_m$ and $\eta>0,\eps>0$, define
\[
\omega_{\eps,\eta,N}(F_m)
:=
\frac{\int_{\R^N}e^{-(\eta-i/\eps)S_N}F_m(u)\,du\,dv}
{\int_{\R^N}e^{-(\eta-i/\eps)S_N}\,du\,dv}.
\]
Then:
\begin{enumerate}[label=(\roman*),leftmargin=1.6em]
\item $\omega_{\eps,\eta,N}(F_m)$ converges as $N\to\infty$.
\item There exists $C_{F_m,\eps,\eta}>0$ such that for $N'>N\ge m$,
\[
|\omega_{\eps,\eta,N'}(F_m)-\omega_{\eps,\eta,N}(F_m)|
\le
C_{F_m,\eps,\eta}\sum_{j=N+1}^{N'}\frac{A_j}{\lambda_j}.
\]
\end{enumerate}
\end{theorem}

\begin{proof}
Integrate each Gaussian tail coordinate:
\[
\int_\R e^{-(\eta-i/\eps)\left(\frac{\lambda_j}{2}+\beta_j(u)\right)t^2}dt
=
\sqrt{\frac{2\pi}{\eta-i/\eps}}\,
\left(\lambda_j+2\beta_j(u)\right)^{-1/2}.
\]
Constants independent of $u$ cancel in the normalized ratio, giving
\[
\omega_{\eps,\eta,N}(F_m)=
\frac{\mathcal N_N(F_m)}{\mathcal D_N},
\]
with
\[
\mathcal N_N(F):=\int_{\R^m}e^{-(\eta-i/\eps)P_m(u)}F(u)\Phi_N(u)\,du,
\]
\[
\Phi_N(u):=\prod_{j=m+1}^N R_j(u),\qquad
R_j(u):=\left(\frac{\lambda_j}{\lambda_j+2\beta_j(u)}\right)^{1/2}\in(0,1].
\]
Now
\[
-\log R_j(u)=\frac12\log\!\left(1+\frac{2\beta_j(u)}{\lambda_j}\right)
\le
\frac{\beta_j(u)}{\lambda_j}
\le
\|u\|^2\frac{A_j}{\lambda_j}.
\]
Hence $\sum_j |\log R_j(u)|<\infty$, so $\Phi_N(u)\to\Phi_\infty(u)\in(0,1]$.
By coercivity of $P_m$ and $|\Phi_N|\le1$:
\[
|\mathcal N_N(F)|\le \|F\|_\infty \int e^{-\eta P_m(u)}du<\infty,
\]
thus dominated convergence gives $\mathcal N_N(F)\to\mathcal N_\infty(F)$ and $\mathcal D_N\to\mathcal D_\infty$.
Assuming $\mathcal D_\infty\neq0$, ratios converge.

For the rate, write $\Phi_{N'}=\Phi_N\Psi_{N,N'}$, $\Psi_{N,N'}:=\prod_{j=N+1}^{N'}R_j$.
Because $0<R_j\le1$,
\[
1-\Psi_{N,N'}\le \sum_{j=N+1}^{N'}(1-R_j).
\]
Set $t_j=2\beta_j/\lambda_j\ge0$.
Since $1-(1+t)^{-1/2}\le t$ for $t\ge0$,
\[
1-R_j(u)\le \frac{2\beta_j(u)}{\lambda_j}
\le 2\|u\|^2\frac{A_j}{\lambda_j}.
\]
Therefore
\[
|\Phi_{N'}(u)-\Phi_N(u)|
\le
2\|u\|^2\sum_{j=N+1}^{N'}\frac{A_j}{\lambda_j}.
\]
Insert this bound in $\mathcal N,\mathcal D$ differences and use
$|e^{-(\eta-i/\eps)P_m}|\le e^{-\eta P_m}$.
Then for $C_1:=2\int e^{-\eta P_m}\|u\|^2du<\infty$:
\[
|\mathcal N_{N'}(F)-\mathcal N_N(F)|
\le
\|F\|_\infty C_1\sum_{j=N+1}^{N'}\frac{A_j}{\lambda_j},
\]
\[
|\mathcal D_{N'}-\mathcal D_N|
\le
C_1\sum_{j=N+1}^{N'}\frac{A_j}{\lambda_j}.
\]
For large $N$, $|\mathcal D_N|\ge d_*>0$, and
\[
\left|\frac{a'}{b'}-\frac{a}{b}\right|
\le
\frac{|a'-a|}{|b'|}+\frac{|a|\,|b'-b|}{|b'||b|}
\]
gives the stated rate. 
\end{proof}

\begin{theorem}[Non-factorized quartic-tail large-$N$ extension]
\label{thm:quartic-tail}
Let
\[
S_N(u,v)
=
P_m(u)+\sum_{j=m+1}^N
\Big(\big(\tfrac{\lambda_j}{2}+\beta_j(u)\big)v_j^2+\gamma_j v_j^4\Big),
\]
with $\lambda_j\ge\lambda_->0$, $\gamma_j\ge\gamma_->0$, coercive $P_m$, and
\[
\beta_j(u)\le A_j\|u\|^2,\qquad A_j\ge0.
\]
For
\[
I_j(b):=\int_\R e^{-c((\lambda_j/2+b)t^2+\gamma_j t^4)}dt,\quad c=\eta-i/\eps,\ b\ge0,
\]
assume:
\begin{enumerate}[label=(Q\arabic*),leftmargin=1.6em]
\item $I_j(b)\neq0$ for all $j,b\ge0$,
\item $\sup_{b\ge0}|\partial_b\log I_j(b)|\le L_j$ and
\[
\sum_{j=m+1}^\infty L_jA_j<\infty.
\]
\end{enumerate}
Then for bounded cylinder observables $F_m$,
\[
\omega_{\eps,\eta,N}(F_m)
:=
\frac{\int e^{-cS_N}F_m(u)\,du\,dv}{\int e^{-cS_N}\,du\,dv}
\]
converges as $N\to\infty$ (if the limiting denominator is nonzero), and satisfies the tail estimate
\[
|\omega_{\eps,\eta,N'}(F_m)-\omega_{\eps,\eta,N}(F_m)|
\le
C_{F_m,\eps,\eta}\sum_{j=N+1}^{N'}L_jA_j.
\]
\end{theorem}

\begin{proof}
Integrate each $v_j$:
\[
\omega_{\eps,\eta,N}(F_m)=
\frac{\int_{\R^m}e^{-cP_m(u)}F_m(u)\Phi_N(u)\,du}
{\int_{\R^m}e^{-cP_m(u)}\Phi_N(u)\,du},
\quad
\Phi_N(u)=\prod_{j=m+1}^N\frac{I_j(\beta_j(u))}{I_j(0)}.
\]
For each $j$,
\[
\left|\log\frac{I_j(\beta_j(u))}{I_j(0)}\right|
=
\left|\int_0^{\beta_j(u)}\partial_b\log I_j(b)\,db\right|
\le
L_jA_j\|u\|^2.
\]
Hence
\[
\sum_{j=m+1}^\infty\left|\log\frac{I_j(\beta_j(u))}{I_j(0)}\right|
\le
\|u\|^2\sum_{j=m+1}^\infty L_jA_j
<\infty,
\]
so $\Phi_N(u)\to\Phi_\infty(u)$ pointwise and
\[
|\Phi_N(u)|\le \exp(B\|u\|^2),\qquad B:=\sum_{j=m+1}^\infty L_jA_j.
\]
Thus
\[
|e^{-cP_m(u)}\Phi_N(u)F_m(u)|
\le
\|F_m\|_\infty e^{-\eta P_m(u)}e^{B\|u\|^2},
\]
integrable by quartic coercivity; dominated convergence yields numerator/denominator limits and ratio convergence.

For the rate, define
\[
\Delta_{N,N'}(u):=\sum_{j=N+1}^{N'}\log\frac{I_j(\beta_j(u))}{I_j(0)},
\quad
|\Delta_{N,N'}(u)|\le \|u\|^2\sum_{j=N+1}^{N'}L_jA_j.
\]
With $\Phi_{N'}=\Phi_N e^{\Delta_{N,N'}}$:
\[
|\Phi_{N'}-\Phi_N|
\le
|\Phi_N|\,|e^{\Delta_{N,N'}}-1|
\le
e^{B\|u\|^2}\,|\Delta_{N,N'}|e^{|\Delta_{N,N'}|}.
\]
This gives
\[
|\Phi_{N'}-\Phi_N|
\le
e^{(B+\tilde B)\|u\|^2}\|u\|^2\sum_{j=N+1}^{N'}L_jA_j,
\]
for a finite $\tilde B$ (tail-sum bound). Integrating against $e^{-\eta P_m}$ gives numerator/denominator Cauchy bounds, and the ratio estimate follows as in Theorem \ref{thm:largen-coupled}. 
\end{proof}

\begin{corollary}[Intrinsic sufficient conditions for Theorem \ref{thm:quartic-tail}]
\label{cor:quartic-intrinsic}
For each $j$, define block moments
\[
\overline M^{(1)}_j:=\sup_{b\ge0}\mathbb E_{\nu_{j,b}}[S_{j,b}],\qquad
\overline M^{(2)}_j:=\sup_{b\ge0}\mathbb E_{\nu_{j,b}}[t^2],
\]
where
\[
\nu_{j,b}(dt):=\frac{e^{-\eta S_{j,b}(t)}}{\int e^{-\eta S_{j,b}}}\,dt,\quad
S_{j,b}(t)=\left(\frac{\lambda_j}{2}+b\right)t^2+\gamma_j t^4.
\]
If
\[
\eps>\sup_j \overline M^{(1)}_j,
\]
and
\[
\sum_{j=m+1}^\infty
A_j\frac{|c|\,\overline M^{(2)}_j}{1-\overline M^{(1)}_j/\eps}
<\infty,
\]
then hypotheses (Q1)--(Q2) in Theorem \ref{thm:quartic-tail} hold with
\[
L_j=
\frac{|c|\,\overline M^{(2)}_j}{1-\overline M^{(1)}_j/\eps}.
\]
\end{corollary}

\begin{proof}
By Theorem \ref{thm:nonvanish} applied to each block $S_{j,b}$:
\[
|I_j(b)|\ge \left(\int e^{-\eta S_{j,b}}\right)\left(1-\frac{\overline M^{(1)}_j}{\eps}\right)>0,
\]
so (Q1) holds.
Also
\[
\partial_b I_j(b)=-c\int t^2 e^{-cS_{j,b}(t)}dt,
\]
thus
\[
\left|\partial_b\log I_j(b)\right|
\le
\frac{|c|\int t^2e^{-\eta S_{j,b}}dt}
{\int e^{-\eta S_{j,b}}dt\,(1-\overline M^{(1)}_j/\eps)}
\le
\frac{|c|\,\overline M^{(2)}_j}{1-\overline M^{(1)}_j/\eps}.
\]
This is (Q2). Summability is exactly the second assumption.
\end{proof}

\section{Partition-Factor Non-Vanishing Bounds}

\begin{theorem}[Moment criteria]
\label{thm:nonvanish}
Let $A_\eta=\int e^{-\eta S(x)}dx\in(0,\infty)$ and
\[
Z_{\eps,\eta}:=\int e^{-(\eta-i/\eps)S(x)}dx
=
A_\eta\,\mathbb E_{\mu_\eta}[e^{iS/\eps}],
\quad
\mu_\eta(dx):=\frac{e^{-\eta S(x)}}{A_\eta}\,dx.
\]
Define
\[
M_1:=\mathbb E_{\mu_\eta}|S|,\qquad M_2:=\mathbb E_{\mu_\eta}(S^2).
\]
Then
\[
|Z_{\eps,\eta}|\ge A_\eta\left(1-\frac{M_1}{\eps}\right),
\]
\[
|Z_{\eps,\eta}|\ge A_\eta\left(1-\frac{M_2}{2\eps^2}\right).
\]
Hence if $\eps>M_1$ or $\eps^2>M_2/2$, then $Z_{\eps,\eta}\neq0$.
\end{theorem}

\begin{proof}
First bound:
\[
\left|\mathbb E[e^{iX}]\right|
=|1+\mathbb E(e^{iX}-1)|
\ge1-\mathbb E|e^{iX}-1|,
\quad X=S/\eps.
\]
Since $|e^{it}-1|\le |t|$,
\[
\left|\mathbb E[e^{iS/\eps}]\right|
\ge1-\frac{M_1}{\eps}.
\]
Multiply by $A_\eta$.

Second bound:
\[
\Re\mathbb E[e^{iS/\eps}]
=\mathbb E[\cos(S/\eps)]
\ge1-\frac{1}{2}\mathbb E[(S/\eps)^2]
=1-\frac{M_2}{2\eps^2}.
\]
Now $|z|\ge \Re z$ gives the inequality for $|Z_{\eps,\eta}|$.
\end{proof}

\section{Observable-Class Extension}

\begin{theorem}[Continuity on Schwartz and weighted Sobolev classes]
\label{thm:observable-extension}
Let
\[
\mathcal I(F)=\int_{\R^d}e^{i\Phi(y)}W(y)F(Ay)\,dy,
\]
with $A\in GL(d,\C)$ and
\[
|W(y)|\le C_0 e^{-c_4\|y\|^4+c_2\|y\|^2},\qquad c_4>0.
\]
Then:
\begin{enumerate}[label=(\roman*),leftmargin=1.6em]
\item for every integer $k>d$, there exists $C_k$ such that
\[
|\mathcal I(F)|\le C_k \sup_x (1+\|x\|)^k|F(x)|,\qquad F\in\mathcal S(\R^d);
\]
\item for every $k>d/2$, there exists $C_k'$ such that
\[
|\mathcal I(F)|\le C_k' \|(1+\|x\|^2)^{k/2}F\|_{L^2},\qquad F\in H^{0,k}.
\]
\end{enumerate}
Consequently, normalized functionals $\omega(F)=\mathcal I(F)/\mathcal I(1)$ (when $\mathcal I(1)\neq0$) extend continuously from Gaussian-polynomial test families to both classes.
\end{theorem}

\begin{proof}
For Schwartz:
\[
|F(Ay)|\le C_A p_k(F)(1+\|y\|)^{-k},
\quad p_k(F):=\sup_x(1+\|x\|)^k|F(x)|.
\]
Hence
\[
|\mathcal I(F)|
\le
C_0C_A p_k(F)\int e^{-c_4\|y\|^4+c_2\|y\|^2}(1+\|y\|)^{-k}dy,
\]
and the integral is finite.

For weighted Sobolev:
\[
|\mathcal I(F)|
\le
\|W(\cdot)(1+\|\cdot\|^2)^{-k/2}\|_{L^2}
\cdot
\|(1+\|y\|^2)^{k/2}F(Ay)\|_{L^2_y}.
\]
The first factor is finite by quartic decay; the second is bounded by $C_A'\|F\|_{H^{0,k}}$ after linear change of variables.
\end{proof}

\section{Schwinger-Dyson and \texorpdfstring{$\tau_\mu$}{tau_mu} Scale Covariance}

\begin{theorem}[Finite-dimensional Schwinger-Dyson identity]
\label{thm:sd}
Let $c=\eta-i/\eps$ and
\[
\mathcal I_c(F):=\int e^{-cS(x)}F(x)\,dx.
\]
Assume integrability and vanishing boundary flux for admissible $F$ and vector field $V$.
Then
\[
\mathcal I_c(V\cdot\nabla S\,F)
=
\frac{1}{c}\mathcal I_c(\nabla\!\cdot(VF)).
\]
If $\mathcal I_c(1)\neq0$, then
\[
\omega_c(V\cdot\nabla S\,F)
=
\frac{1}{c}\omega_c(\nabla\!\cdot(VF)).
\]
In particular, for constant $V=e_i$ and $F\equiv1$:
\[
\omega_c(\partial_i S)=0.
\]
\end{theorem}

\begin{proof}
\[
0=\int \nabla\!\cdot(e^{-cS}VF)\,dx
=\int e^{-cS}(\nabla\!\cdot(VF)-c\,V\cdot\nabla S\,F)\,dx.
\]
Rearrange, then divide by $\mathcal I_c(1)$ for the normalized form.
\end{proof}

\begin{theorem}[Exact \texorpdfstring{$\tau_\mu$}{tau_mu} covariance]
\label{thm:tau}
For
\[
\omega_{\kappa,\eta,h}(F):=
\frac{\int e^{-(\eta-i/h)\kappa S(x)}F(x)\,dx}
{\int e^{-(\eta-i/h)\kappa S(x)}\,dx},
\]
define
\[
\tau_\mu:(\kappa,\eta,h)\mapsto(\mu\kappa,\eta/\mu,\mu h),\qquad \mu>0.
\]
Then
\[
\omega_{\kappa,\eta,h}(F)=\omega_{\tau_\mu(\kappa,\eta,h)}(F).
\]
\end{theorem}

\begin{proof}
Directly,
\[
\left(\frac{\eta}{\mu}-\frac{i}{\mu h}\right)(\mu\kappa)
=(\eta-i/h)\kappa.
\]
Hence numerator and denominator kernels are unchanged.
\end{proof}

\section{Dependency Chain}

The proofs in this manuscript now close in the following order:
\begin{enumerate}[leftmargin=1.6em]
\item Sections \ref{sec:projective}--\ref{sec:channels}: core scoped Claim 1 closure.
\item Theorems \ref{thm:largen-coupled}, \ref{thm:quartic-tail}, and Corollary \ref{cor:quartic-intrinsic}: large-$N$ coupled extensions (Gaussian-tail rate, non-factorized quartic-tail class, and intrinsic moment-based sufficient conditions).
\item Theorem \ref{thm:nonvanish}: explicit non-vanishing criteria for partition factors.
\item Theorem \ref{thm:observable-extension}: observable-class extension to Schwartz/Sobolev.
\item Theorems \ref{thm:sd} and \ref{thm:tau}: Schwinger-Dyson identities and exact scale-flow covariance.
\end{enumerate}

\section{Conclusion}

This manuscript now gives a theorem-grade chain from the static/finite-dimensional variational-delta picture to a broad scoped family of oscillatory states including:
\begin{enumerate}[leftmargin=1.6em]
\item exact projective closure and de-regularization,
\item explicit semiclassical channel expansion,
\item large-$N$ mode-coupled convergence with quantitative tail control (Gaussian-tail and non-factorized quartic-tail classes), with intrinsic moment criteria for quartic-tail hypotheses,
\item non-vanishing control for normalization factors,
\item extension to large observable classes,
\item Schwinger-Dyson identities and scale-flow covariance.
\end{enumerate}

Open frontier (outside this closure): full interacting, non-factorized, genuinely field-theoretic continuum limits with uniform renormalization control beyond the scoped polynomial block-tail class.

\begin{thebibliography}{99}

\bibitem{feynman1948}
R. P. Feynman,
\emph{Space-Time Approach to Non-Relativistic Quantum Mechanics},
Rev. Mod. Phys. \textbf{20} (1948), 367--387.
DOI: \href{https://doi.org/10.1103/RevModPhys.20.367}{10.1103/RevModPhys.20.367}.

\bibitem{hormander1}
L. H\"ormander,
\emph{The Analysis of Linear Partial Differential Operators I:
Distribution Theory and Fourier Analysis},
Springer, 2nd ed., 2003.
DOI: \href{https://doi.org/10.1007/978-3-642-61497-2}{10.1007/978-3-642-61497-2}.

\bibitem{gelfandshilov1}
I. M. Gel'fand and G. E. Shilov,
\emph{Generalized Functions, Vol. 1: Properties and Operations},
AMS Chelsea, 1964.
DOI: \href{https://doi.org/10.1090/chel/377}{10.1090/chel/377}.

\bibitem{albeverio2008}
S. Albeverio, R. J. H\o egh-Krohn, and S. Mazzucchi,
\emph{Mathematical Theory of Feynman Path Integrals: An Introduction},
Lecture Notes in Mathematics 523, 2nd ed., Springer, 2008.
DOI: \href{https://doi.org/10.1007/978-3-540-76956-9}{10.1007/978-3-540-76956-9}.

\bibitem{connes1994}
A. Connes,
\emph{Noncommutative Geometry},
Academic Press, 1994.
ISBN: 978-0-12-185860-5.

\bibitem{landsman1998}
N. P. Landsman,
\emph{Mathematical Topics Between Classical and Quantum Mechanics},
Springer Monographs in Mathematics, Springer, 1998.
DOI: \href{https://doi.org/10.1007/978-1-4612-1680-3}{10.1007/978-1-4612-1680-3}.

\bibitem{landsman1999}
N. P. Landsman,
\emph{Lie Groupoid C*-Algebras and Weyl Quantization},
Commun. Math. Phys. \textbf{206} (1999), 367--381.
DOI: \href{https://doi.org/10.1007/s002200050709}{10.1007/s002200050709}.

\bibitem{landsmanramazan2001}
N. P. Landsman and B. Ramazan,
\emph{Quantization of Poisson algebras associated to Lie algebroids},
in \emph{Groupoids in Analysis, Geometry, and Physics},
Contemporary Mathematics \textbf{282}, AMS, 2001, 159--192.
DOI: \href{https://doi.org/10.1090/conm/282}{10.1090/conm/282}.

\end{thebibliography}

\end{document}
