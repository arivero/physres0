\documentclass[11pt]{article}
\usepackage[a4paper,margin=1in]{geometry}
\usepackage{amsmath,amssymb,amsthm,mathtools}

\newtheorem{proposition}{Proposition}
\newtheorem{remark}{Remark}

\title{Claim 1 Note: Variational Delta from Static Integrals to QM and QFT}
\author{}
\date{2026-02-08}

\begin{document}
\maketitle

\begin{abstract}
This note organizes Claim 1 into a three-level ladder:
static finite-dimensional oscillatory integrals, quantum mechanics in time, and quantum field theory in spacetime.
At each level, the same object appears: a Dirac delta of the first variation, supported on extrema of the corresponding action.
The static level is theorem-grade; the full infinite-dimensional bridge to continuum path integrals remains conjectural.
\end{abstract}

\section{Notation and Scope}

We use the shorthand
\[
\delta(\partial S)
\]
for ``delta of first variation'' (or gradient in finite dimensions), i.e. a Dirac concentration on critical points.
This is \emph{not} the distribution derivative $\partial_x\delta$ (often denoted $\delta'$): it is composition/pullback by the map $\partial S$.

At the same time, both belong the point-supported-distribution framework:
in one dimension, any point-supported distribution is a finite sum
\[
\sum_{m=0}^N c_m\,\delta^{(m)}.
\]
Under dilation $x\mapsto \lambda x$ (\(\lambda>0\)),
\[
\delta^{(m)}(\lambda x)=\lambda^{-m-1}\delta^{(m)}(x),
\]
so different derivatives of $\delta$ carry different scaling weights, i.e. multiple scaling modes/fixed channels.

\section{Level 0: Static Oscillatory Integral}

Let $f \in C^\infty(\mathbb{R})$ and $O \in C_c^\infty(\mathbb{R})$. Define
\[
A_\varepsilon(O) := \varepsilon^{-1/2}\int_{\mathbb{R}} e^{\frac{i}{\varepsilon}f(x)}\,O(x)\,dx,
\qquad \varepsilon>0.
\]

\begin{proposition}[Single nondegenerate critical point]
Assume $f'(x_0)=0$, $f''(x_0)\neq 0$, and $x_0$ is the unique critical point.
Then, as $\varepsilon\to 0^+$,
\[
|A_\varepsilon(O)|^2
=
2\pi\,\frac{|O(x_0)|^2}{|f''(x_0)|}
+o(1),
\]
up to Fourier-normalization convention.
\end{proposition}

\begin{proof}[Proof sketch]
Apply stationary phase to $A_\varepsilon$:
\[
\int e^{\frac{i}{\varepsilon}f(x)}O(x)\,dx
\sim
e^{\frac{i}{\varepsilon}f(x_0)}e^{i\frac{\pi}{4}\operatorname{sgn}(f''(x_0))}
\sqrt{\frac{2\pi\varepsilon}{|f''(x_0)|}}\,O(x_0).
\]
Multiply by $\varepsilon^{-1/2}$, then take modulus squared.
\end{proof}

Distributionally, this is the finite-dimensional template
\[
|A_\varepsilon(O)|^2 \to 2\pi\,\langle \delta(f'), |O|^2\rangle.
\]

\begin{remark}[Multiple critical points]
With several nondegenerate critical points, cross terms carry phases
$e^{i(f(x_i)-f(x_j))/\varepsilon}$.
Pointwise limits of $|A_\varepsilon|^2$ can fail without averaging/weak-limit prescriptions.
\end{remark}

\section{Level 1: Quantum Mechanics (Action in Time)}

For paths $\phi:[t_0,t_1]\to M$ and action
\[
S[\phi]=\int_{t_0}^{t_1}L(\phi,\dot\phi)\,dt,
\]
the critical-point equation is Euler--Lagrange:
\[
\frac{\delta S}{\delta \phi}=0.
\]
The direct analogue of $\delta(f')$ is
\[
\Delta_{\mathrm{QM}}:=\delta\!\left(\frac{\delta S}{\delta \phi}\right),
\]
formally supported on classical trajectories.

\subsection*{Discrete approximation}
Time slicing yields finite-dimensional variables $(q_1,\dots,q_N)$ and discrete action $S_N(q_1,\dots,q_N)$.
Then
\[
\delta\!\left(\frac{\delta S}{\delta \phi}\right)
\rightsquigarrow
\delta(\nabla S_N),
\]
where $\nabla S_N=0$ are the discrete Euler--Lagrange equations.
This is the exact finite-dimensional support statement at each fixed $N$.

\section{Level 2: Quantum Field Theory (Action in Spacetime)}

For a field $\Phi$ in $D$ dimensions,
\[
S[\Phi]=\int \mathcal{L}(\Phi,\partial\Phi)\,d^Dx,
\qquad
\frac{\delta S}{\delta \Phi}=0.
\]
The corresponding object is
\[
\Delta_{\mathrm{QFT}}:=\delta\!\left(\frac{\delta S}{\delta \Phi}\right),
\]
formally concentrated on classical field equations.

\subsection*{Lattice regularization}
On a lattice with finitely many variables $\Phi_1,\dots,\Phi_N$,
\[
S[\Phi] \rightsquigarrow S_N(\Phi_1,\dots,\Phi_N),\qquad
\Delta_{\mathrm{QFT}} \rightsquigarrow \delta(\nabla S_N),
\]
again making the ``support on extrema'' statement exact at fixed cutoff.

\section{Amplitude and Geometric Representation}

The oscillatory probability-amplitude viewpoint is:
\begin{itemize}
\item amplitudes can be represented geometrically as \(1/2\)-density-level objects,
\item modulus square produces density-level concentration on critical sets.
\end{itemize}

In finite dimensions this is standard stationary phase/distribution theory.
The open step is to prove, with full control, the continuum limit from finite-dimensional regularizations to a rigorous infinite-dimensional object compatible with tangent-groupoid composition and renormalized QFT limits.

\section{Current Program Status (Date-Anchored)}

As of 2026-02-09, the ladder has the following state:
\begin{itemize}
\item \textbf{Statics:} theorem-grade closure in nondegenerate stationary-phase regimes (probability-amplitude to Born-density map).
\item \textbf{Dynamics:} scoped consistency chain established for time-sliced transition amplitudes with explicit refinement/de-regularization/SD assumptions.
\item \textbf{Fields:} \(d=2\) interacting ultralocal closure is established; \(d=3\) has a renormalized finite-volume bound channel and a scoped continuum-branch candidate; \(d=4\) remains frontier-sensitive and requires explicit renormalization/nontriviality input.
\end{itemize}

\section{Summary}

The core structural claim survives escalation:
\[
\text{static } \delta(\nabla S)
\;\longrightarrow\;
\text{QM } \delta\!\left(\frac{\delta S}{\delta\phi}\right)
\;\longrightarrow\;
\text{QFT } \delta\!\left(\frac{\delta S}{\delta\Phi}\right).
\]
At every stage, the delta of first variation localizes on extrema of the corresponding action.

\end{document}
