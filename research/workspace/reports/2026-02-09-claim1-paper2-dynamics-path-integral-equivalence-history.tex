\documentclass[11pt]{article}
\usepackage[a4paper,margin=1in]{geometry}
\usepackage{amsmath,amssymb,amsthm,mathtools}
\usepackage{hyperref}
\usepackage{upquote}

\newtheorem{theorem}{Theorem}
\newtheorem{proposition}{Proposition}
\newtheorem{corollary}{Corollary}
\newtheorem{remark}{Remark}

\title{Dynamic Variational Consistency and Scoped Equivalence to the Path Integral\\
with Historical Discussion}
\author{}
\date{2026-02-09}

\begin{document}
\maketitle

\begin{abstract}
This paper develops a dynamic consistency chain for \(0+1\)-dimensional actions:
time-sliced transition amplitudes induce normalized cylinder states that are stable under refinement, admit \(\eta\to0^+\) de-regularization, satisfy Schwinger--Dyson identities, and are invariant under the \(c\)-preserving flow \(\tau_\mu\).
Under explicit assumptions, these statements imply scoped equivalence to the path-integral formalism for the same model class and boundary data.
A dedicated historical section records the Dirac (1933) \(\to\) Feynman (1948) \(\to\) Wilson--Kogut (1974) lineage.
\end{abstract}

\section{Scope}

\subsection*{In-scope claim}
The claim proved here is scoped dynamic equivalence:
\begin{enumerate}
\item construction of a consistent continuum cylinder state from time slicing,
\item persistence of Schwinger--Dyson structure in the limit,
\item invariance along \(c\)-preserving reparameterizations,
\item identification of that limit with the path-integral functional in the same scoped class.
\end{enumerate}

\subsection*{Out of scope}
\begin{enumerate}
\item full interacting field-theory closure in \(d=4\),
\item model-independent nonperturbative global path-integral existence,
\item scattering/unitarity claims beyond the scoped construction.
\end{enumerate}

\section{Dynamic setup}

Let
\[
S[q]=\int_{t_i}^{t_f}L(q,\dot q,t)\,dt.
\]
At time-slicing resolution \(N\), write the discretized action \(S_N\) and define a regularized transition-amplitude channel
\(x_0,x_N\) fixed boundary data and \(N-1\) interior integrations:
\[
\mathcal A_{\varepsilon,\varepsilon_0,N}(O)
:=
\int_{\mathbb R^{N-1}} \varepsilon^{-(N-1)/2}
\exp\!\left(\frac{i}{\varepsilon}\varepsilon_0\sum_{k=0}^{N-1}L_{\varepsilon_0}\!\left[x_k,\frac{x_{k+1}-x_k}{\varepsilon_0}\right]\right)
O[x_\bullet]\prod_{k=1}^{N-1}dx_k.
\]

For the normalized state channel in the dressed notation, use
\[
\omega_{c,N}(F):=
\frac{\int_{\mathbb R^N}e^{-cS_N(x)}F(x)\,dx}
{\int_{\mathbb R^N}e^{-cS_N(x)}\,dx},
\qquad c=(\eta-i/h)\kappa,\quad \Re c>0.
\]

\section{Dynamic consistency theorem chain}

\subsection*{Assumptions}
Assume the following on a compact \(c\)-domain \(K\):
\begin{enumerate}
\item[(D1)] \textbf{Projective compatibility}: cylinder observables are consistently lifted along \(N\mapsto N+1\).
\item[(D2)] \textbf{Denominator non-vanishing}: \(\int e^{-cS_N}\neq 0\) for all large \(N\), uniformly in \(c\in K\).
\item[(D3)] \textbf{Uniform Cauchy tail}: for each cylinder \(F_m\),
\[
|\omega_{c,N'}(F_m)-\omega_{c,N}(F_m)|\le C_{F_m,K}\tau_N,\qquad \tau_N\to0.
\]
\item[(D4)] \textbf{Finite-\(N\) Schwinger--Dyson identity}: for admissible \(\psi\),
\[
\omega_{c,N}(\partial_i\psi)=c\,\omega_{c,N}(\psi\,\partial_iS_N).
\]
\item[(D5)] \textbf{\(c\)-invariance along \(\tau_\mu\)}:
\[
\tau_\mu:(\kappa,\eta,h)\mapsto(\mu\kappa,\eta/\mu,\mu h),\quad \mu>0,
\]
preserves \(c\), hence preserves the finite-\(N\) kernels.
\item[(D6)] \textbf{De-regularization pass-through}: for observables in the working class, the one-sided limit \(\eta\to0^+\) exists and commutes with the \(N\to\infty\) limit in the stated channel.
\end{enumerate}

\begin{theorem}[Scoped dynamic consistency]
Under \((D1)\)--\((D6)\):
\begin{enumerate}
\item \(\omega_{c,N}(F_m)\to\omega_c(F_m)\) for each cylinder \(F_m\), uniformly on compact \(K\subset\{c:\Re c>0\}\).
\item \(\omega_c\) is invariant along \(\tau_\mu\)-orbits (depends on parameters only through \(c\)).
\item Schwinger--Dyson identities pass to the limit:
\[
\omega_c(\partial_i\psi)=c\,\omega_c(\psi\,G_i),
\]
where \(G_i\) is the cylinder-pairing limit of \(\partial_iS_N\).
\item The de-regularized state \(\omega_{-i\kappa/h}\) exists (at fixed \(\kappa\)) in the scoped observable class.
\end{enumerate}
\end{theorem}

\begin{proof}[Proof sketch]
\begin{enumerate}
\item (D3) + (D2) give convergence of normalized cylinder expectations.
\item (D5) gives exact finite-\(N\) kernel invariance; pass to \(N\to\infty\).
\item (D4) + uniform pairing control imply SD pass-through in the limit.
\item (D6) gives existence/commutation of the \(\eta\to0^+\) channel in scope.
\end{enumerate}
\end{proof}

\begin{corollary}[Scoped path-integral equivalence]
Define the scoped path-integral functional by the refinement limit of the normalized time-sliced transition-amplitude channel under \((D1)\)--\((D6)\). Then this functional equals \(\omega_c\) on the scoped cylinder observable class.
\end{corollary}

\begin{remark}
The corollary is an equivalence \emph{in the scoped class} where all convergence and non-vanishing gates are discharged. It is not a claim of global interacting QFT closure.
\end{remark}

\section{Historical discussion (required)}

\subsection*{Dirac (1933)}
Dirac's 1933 analysis emphasized the role of an exponential phase factor in quantum transitions, providing the conceptual seed for amplitude-level composition rules.

\subsection*{Feynman (1948)}
Feynman recast quantum mechanics via time slicing and action-phase summation over histories, making transition-amplitude composition over intermediate configurations explicit.

\subsection*{Wilson--Kogut (1974)}
Wilson--Kogut RG language reframed refinement as scale-flow control, clarifying why continuum claims require explicit fixed-point and convergence gates rather than formal passage to finer slices.

\subsection*{Current theoretical framing}
The present program combines these threads:
\begin{enumerate}
\item transition-amplitude composition at finite slicing,
\item explicit convergence/non-vanishing gates,
\item \(c\)-invariant scale-flow covariance and Schwinger--Dyson persistence,
\item geometric \(1/2\)-density language only where kernel-bundle structure is explicit.
\end{enumerate}

\section{Validation contract (Goal 1B)}

\subsection*{Assumptions}
Model class, boundary data, regulator net, observable class, and normalization domain are explicit in \((D1)\)--\((D6)\).

\subsection*{Units and dimensions check}
\begin{enumerate}
\item each action contribution has action dimensions,
\item phase \(S/\varepsilon\) is dimensionless,
\item slicing normalization factors are tracked at each \(N\).
\end{enumerate}

\subsection*{Symmetry and conservation checks}
\begin{enumerate}
\item boundary-condition consistency under slicing/refinement,
\item finite-\(N\) variational symmetry identities used before continuum passage,
\item conservation-law checks in scoped model channels where relevant.
\end{enumerate}

\subsection*{Independent cross-check paths}
\begin{enumerate}
\item analytic finite-dimensional SD and scale-flow covariance derivation channels,
\item numerical diagnostics:
\begin{itemize}
\item \texttt{python3.12 research/workspace/simulations/claim1\_cylinder\_gaussian\_toy.py},
\item \texttt{python3.12 research/workspace/simulations/claim1\_continuum\_cauchy\_diagnostics.py},
\item \texttt{python3.12 research/workspace/simulations/claim1\_fd\_schwinger\_dyson\_check.py},
\item \texttt{python3.12 research/workspace/simulations/claim1\_scale\_flow\_covariance\_check.py},
\item \texttt{python3.12 research/workspace/simulations/claim1\_groupoid\_tau\_sd\_dependency\_check.py}.
\end{itemize}
\item optional formal-companion modules (inequality/derivative backbone):
\begin{itemize}
\item \texttt{research/workspace/proofs/Claim1lean/CovarianceDerivative.lean},
\item \texttt{research/workspace/proofs/Claim1lean/RatioStateDerivativeBound.lean},
\item \texttt{research/workspace/proofs/Claim1lean/RatioStateIncrementBound.lean},
\item \texttt{research/workspace/proofs/Claim1lean/FiniteExponentialIncrementBound.lean},
\item \texttt{research/workspace/proofs/Claim1lean/FiniteExponentialRegularity.lean}.
\end{itemize}
\end{enumerate}

\subsection*{Confidence statement}
The result is theorem-grade in the scoped dynamic class under \((D1)\)--\((D6)\). Any statement outside those gates must be marked unverified.

\section{Reproducibility metadata}

\begin{itemize}
\item build toolchain tested here: \texttt{/Library/TeX/texbin/pdflatex} (TeX Live 2025),
\item safe build script:
\texttt{\textasciitilde/.codex/skills/pdflatex-safe-build/scripts/build\_pdflatex\_safe.sh},
\item date anchor: 2026-02-09 (US).
\end{itemize}

\section{Conclusion}

Dynamic variational consistency is presented here as a self-contained theorem chain:
refinement stability, denominator control, SD persistence, and de-regularization control.
Within those assumptions, scoped equivalence to the path-integral formalism is explicit and historically motivated.
As of 2026-02-09 (US date), this dynamic component is locked at scoped theorem-grade level,
while field-level closure remains an independent Gate G2/G3 program handled in Paper 3.

\end{document}
