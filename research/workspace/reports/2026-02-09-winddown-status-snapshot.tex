\documentclass[11pt]{article}
\usepackage[a4paper,margin=1in]{geometry}
\usepackage{amsmath,amssymb}
\usepackage{hyperref}

\title{Wind-Down Status Snapshot:\\
Learned Results, Open Proof Gates, and Handoff Queue}
\author{}
\date{2026-02-09}

\begin{document}
\maketitle

\begin{abstract}
\textbf{Superseded.}  This snapshot was written on 2026-02-09 when the $d{=}3$
branch was at AN-34A and several weekly tasks had not yet been executed.  It is
now superseded by \texttt{2026-02-13-final-closure-synthesis.tex}, which
reflects the completed state through AN-42, weekly tasks A--F, frozen scores
($\bar{S}=9.19$), and explicitly deferred future-work gates.  This file is
retained for historical reference only.
\end{abstract}

\section{Corpus and Method}

The status pass scanned markdown artifacts under the active workspace
(\texttt{rg --files -g '*.md'}), then extracted state-bearing signals
(\texttt{done}, \texttt{closed}, \texttt{proved}, \texttt{queued},
\texttt{frontier}) and checked canonical paper and audit sheets.

\section{What Is Learned}

\subsection*{Goal 1 Papers}
\begin{enumerate}
\item Paper 1 (statics): scoped theorem-grade static variational consistency and
static measurement-layer equivalence are documented in
\texttt{2026-02-09-claim1-paper1-static-amplitude-qm-equivalence.md}.
\item Paper 2 (dynamics): scoped theorem-grade dynamic consistency and
path-integral-equivalence chain (with historical section) are documented in
\texttt{2026-02-09-claim1-paper2-dynamics-path-integral-equivalence-history.md}.
\item Paper 3 (fields): the dimension-indexed program has progressed through
AN-34A in a scoped \(d=3\) branch, including weighted-local and graph-decay nonlocal
lifts with explicit denominator-rate bookkeeping and a first-principles shell-tail
rate transmutation lane; Lean wrappers now package the exhaustion/regularization
commuting-limit interface.
\end{enumerate}

\subsection*{Top-10 Claim Snapshot}
Using \texttt{2026-02-08-top10-claim-audit.md}:
\begin{enumerate}
\item strong scoped closure is already available for Claims 2, 3, 4, 5, 6, 7, 10,
\item Claim 8 has explicit unresolved high-dimensional rotating sectors,
\item Claim 9 has screened-Abelian theorem closure and scoped non-Abelian transfer lanes,
\item Claim 1 has high scoped maturity with explicit global interacting gap.
\end{enumerate}

\section{Open Proof Gates}

\subsection*{Claim 1 Frontier}
\begin{enumerate}
\item Next (field packaging): exhibit concrete exhaustion/regularization envelopes
for the weighted-local graph-decay nonlocal channels and wire them into the
AN-33L-C commuting-limit wrapper so the field-side statements can invoke the Lean
wrapper without hidden hypotheses.
\item Full \(d=4\) interacting closure still requires explicit G2/G3 completion
(continuum existence and reconstruction gates).
\end{enumerate}

\subsection*{Claim 9 Frontier}
\begin{enumerate}
\item Beyond-window transfer control needs first-principles completion and full
propagation into paper and audit lanes.
\item A new AF draft lane is now recorded in
\texttt{2026-02-09-claim9-nonabelian-firstprinciples-transfer-clustering.md},
with executable diagnostic scaffold
\texttt{claim9\_nonabelian\_first\_principles\_transfer\_check.py}.
\item Dynamical-matter string-breaking crossover remains at program level.
\end{enumerate}

\subsection*{Claim 8 Frontier}
\begin{enumerate}
\item Unresolved \(D\ge 6\) multi-spin rotating sectors remain explicit in the
current theorem map.
\end{enumerate}

\section{Immediate Handoff Queue}

\begin{enumerate}
\item Integrate AF lane into Claim 9 manuscript and audit only after validating
the AF diagnostic and consistency checks in the same pass.
\item Start a dedicated theorem sheet for dynamical-matter string-breaking with
explicit \((G,D,N_f)\)-tagged assumptions and extraction regime.
\item Execute the field-side envelope instantiation and commuting-limit wiring
step, then immediately synchronize Paper 3 under the existing paper update trigger rules.
\end{enumerate}

\section{Validation Contract}

\subsection*{Assumptions}
\begin{enumerate}
\item this report is a status synthesis, not a new theorem derivation,
\item all closure/open labels are inherited from tracked notes and reports.
\end{enumerate}

\subsection*{Units and Dimensions}
\begin{enumerate}
\item no new dimensional claim is introduced in this status document,
\item unit-sensitive statements are delegated to the cited theorem notes.
\end{enumerate}

\subsection*{Symmetry/Conservation}
\begin{enumerate}
\item no new symmetry or conservation claim is asserted here.
\end{enumerate}

\subsection*{Independent Cross-Check Path}
\begin{enumerate}
\item run markdown corpus scan commands,
\item compare extracted statuses with the audit and paper tracks listed above.
\end{enumerate}

\subsection*{Confidence}
High confidence for status extraction consistency. Medium confidence for frontier
ordering, since ordering depends on strategic preference between Claim 1 and
Claim 9 proof acceleration.

\section{Reproducibility Metadata}

\begin{enumerate}
\item date anchor: 2026-02-09 (US),
\item shell environment: \texttt{zsh},
\item TeX build toolchain in this workspace:
\texttt{/Library/TeX/texbin/pdflatex} (TeX Live 2025) via safe wrapper script.
\end{enumerate}

\end{document}
