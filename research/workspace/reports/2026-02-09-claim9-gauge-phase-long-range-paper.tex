\documentclass[11pt]{article}
\usepackage[a4paper,margin=1in]{geometry}
\usepackage{amsmath,amssymb,amsthm,mathtools}
\usepackage{hyperref}
\usepackage{upquote}

\newtheorem{theorem}{Theorem}
\newtheorem{proposition}{Proposition}
\newtheorem{corollary}{Corollary}
\newtheorem{remark}{Remark}

\title{Phase-Resolved Long-Range Gauge Potentials:\\
Screened-Abelian Closure and Confining/Coulomb Program}
\author{}
\date{2026-02-09}

\begin{document}
\maketitle

\begin{abstract}
This paper isolates Claim 9 into assumption-explicit sectors instead of a single
``generic gauge'' statement. We provide a theorem-grade closure for the screened
Abelian branch in arbitrary spacetime dimension, including the exact Yukawa kernel
and its large-distance asymptotic. We also state phase-conditional propositions for
massless Coulomb and confining/string-breaking branches, and record the remaining
rigor gaps in the non-Abelian sector.
\end{abstract}

\section{Scope}

\subsection*{In-scope claim}
\begin{enumerate}
\item phase-conditioned static potential classification by Wilson-loop asymptotics,
\item theorem-grade screened-Abelian branch with explicit dimension dependence,
\item explicit open-gap statement for non-Abelian confinement and dynamical-matter crossover rigor.
\end{enumerate}

\subsection*{Out of scope}
\begin{enumerate}
\item universal phase diagram across all gauge groups and matter contents,
\item claiming non-Abelian confining closure from Abelian screened estimates.
\end{enumerate}

\section{Setup}

Let \(G\) be the gauge group, \(D\) spacetime dimension, and \(n=D-1\) spatial dimension.
Define static potential by rectangular Wilson loops:
\[
\langle W(r,T)\rangle \sim e^{-V(r)T},\qquad T\to\infty.
\]
Long-range classification concerns asymptotics of \(V(r)\) as \(r\to\infty\).

\subsection*{Goal-9 Dependency Declaration}
We treat Claim 9 as
\[
\mathrm{Goal9}(G,D;\text{phase},\text{matter}).
\]
No statement is promoted unless both dependencies are explicit:
\begin{enumerate}
\item gauge-group/model dependency \(G\) (for example \(U(1)\), \(SU(N)\), \(SU(N)\)+fundamental matter),
\item spacetime-dimension dependency \(D\).
\end{enumerate}

\subsection*{Dependency Matrix (Current Program State)}
\begin{center}
\begin{tabular}{|l|l|l|l|}
\hline
\((G,\text{matter},\text{phase})\) & \(D\)-dependence & Asymptotic claim & Status \\
\hline
\((U(1),\ \text{none},\ \text{Coulomb})\) & explicit in all \(D\) & \(r,\log r,r^{3-D}\) & proposition \\
\hline
\((U(1),\ \text{Higgs},\ \text{screened})\) & explicit in all \(D\) & \(r^{-(D-2)/2}e^{-mr}\) & theorem \\
\hline
\((SU(N),\ \text{none},\ \text{confining})\) & dimension-tagged, non-uniform rigor & linear regime \(\sigma r\) & program \\
\hline
\((SU(N),\ N_f>0,\ \text{string-breaking})\) & dimension-tagged, model dependent & linear-to-saturation crossover & program \\
\hline
\end{tabular}
\end{center}

\section{Phase-Conditioned Statements}

\begin{proposition}[Massless Coulomb-class sector \((G=U(1),D)\)]
Assume gauge group \(G=U(1)\), Coulomb phase, and a massless unscreened gauge mode in \(n\) dimensions.
Then
\[
V_{\mathrm{Coul}}(r)\propto g^2 C\,\Phi_n(r),
\]
where \(\Phi_n\) is the Laplacian Green kernel and
\[
\Phi_n(r)\sim
\begin{cases}
r, & n=1\ (D=2),\\
\log r, & n=2\ (D=3),\\
r^{2-n}, & n>2\ (D\ge4).
\end{cases}
\]
\end{proposition}

\begin{theorem}[Screened-Abelian Yukawa branch \((G=U(1),m>0,D)\)]
Let \(G=U(1)\), \(m>0\), \(n=D-1\ge1\), and
\[
(-\Delta+m^2)G_{n,m}=\delta_0
\]
in \(\mathbb{R}^n\), with \(G_{n,m}(r)\to0\) as \(r\to\infty\). Then
\[
G_{n,m}(r)=\frac{1}{(2\pi)^{n/2}}\left(\frac{m}{r}\right)^\nu K_\nu(mr),
\qquad
\nu=\frac n2-1,
\]
and as \(r\to\infty\),
\[
G_{n,m}(r)=
\frac{1}{2(2\pi)^{(n-1)/2}}
m^{(n-3)/2}r^{-(n-1)/2}e^{-mr}
\left(1+O(r^{-1})\right).
\]
For static charges \(q_1,q_2\), \(V(r)=q_1q_2G_{n,m}(r)\), hence
\[
V(r)\sim r^{-(D-2)/2}e^{-mr},
\]
so the inter-source term decays exponentially and the large-\(r\) energy saturates
to an \(r\)-independent baseline.
\end{theorem}

\begin{proof}[Proof sketch]
Use Fourier representation
\[
G_{n,m}(x)=\frac{1}{(2\pi)^n}\int_{\mathbb R^n}\frac{e^{ik\cdot x}}{|k|^2+m^2}\,dk,
\]
radial reduction, and the standard Hankel/Bessel identity to obtain the exact kernel.
Then apply \(K_\nu(z)\sim \sqrt{\pi/(2z)}\,e^{-z}(1+O(z^{-1}))\) as \(z\to+\infty\).
The exponential factor yields saturation of the interaction contribution.
\end{proof}

\begin{proposition}[Confining area-law sector \((G=SU(N),N\ge2,D)\)]
Assume gauge group \(G=SU(N)\) with \(N\ge2\), in a pure-gauge confining regime with
\[
\langle W(r,T)\rangle \sim e^{-\sigma rT},\qquad \sigma>0.
\]
Then the static potential obeys
\[
V_{\mathrm{conf}}(r)\sim \sigma r
\]
in the corresponding large-\(r\) regime.
\end{proposition}

\begin{proposition}[Dynamical-fundamental matter crossover \((G=SU(N),N_f>0,D)\)]
Assume gauge group \(G=SU(N)\) with \(N_f>0\) dynamical fundamental flavors and pair creation/string breaking dynamically allowed.
Then one expects:
\begin{enumerate}
\item approximately linear behavior \(V(r)\sim \sigma r\) at intermediate \(r\),
\item crossover to saturation at sufficiently large \(r\).
\end{enumerate}
\end{proposition}

\begin{corollary}[Claim 9 status in this paper]
Within the screened-Abelian class \((G=U(1),m>0)\), Claim 9 is theorem-closed
with explicit \(D\)-dependence.
The remaining open rigor gap is non-Abelian: deriving confining and string-breaking
crossover statements with fully explicit \((G,D)\)-tagged hypotheses and continuum control in a fixed model class.
\end{corollary}

\section{Literature Anchors}

\begin{enumerate}
\item Wilson and Kogut (1974), RG/phase framing:
\href{https://doi.org/10.1016/0370-1573(74)90023-4}{doi:10.1016/0370-1573(74)90023-4}.
\item Osterwalder and Schrader I (1973), Euclidean axioms:
\href{https://doi.org/10.1007/BF01645738}{doi:10.1007/BF01645738}.
\item Osterwalder and Schrader II (1975), reconstruction:
\href{https://doi.org/10.1007/BF01608978}{doi:10.1007/BF01608978}.
\item Fradkin and Shenker (1979), phase continuity in lattice gauge-Higgs systems:
\href{https://doi.org/10.1103/PhysRevD.19.3682}{doi:10.1103/PhysRevD.19.3682}.
\end{enumerate}

\section{Validation Contract}

\subsection*{Assumptions}
\begin{enumerate}
\item each statement is conditioned on explicit \((G,D)\), phase, and matter tags,
\item screened-Abelian closure uses \(m>0\) and linearized static kernel framework,
\item non-Abelian claims are kept proposition-level unless theorem hypotheses are fully discharged.
\end{enumerate}

\subsection*{Units and dimensions check}
\begin{enumerate}
\item \(mr\) is dimensionless in Yukawa factors,
\item dimensional prefactors in \(G_{n,m}\) match the \(n=D-1\) Green-kernel scaling.
\end{enumerate}

\subsection*{Independent cross-check paths}
\begin{enumerate}
\item analytic Green-kernel and asymptotic derivation (this paper),
\item executable checks:
\begin{itemize}
\item \texttt{python3.12 research/workspace/simulations/claim9\_phase\_longrange\_table.py},
\item \texttt{python3.12 research/workspace/simulations/claim9\_model\_class\_table.py},
\item \texttt{python3.12 research/workspace/simulations/claim9\_abelian\_screened\_asymptotic\_check.py}.
\end{itemize}
\end{enumerate}

\subsection*{Confidence statement}
Screened-Abelian long-range behavior is theorem-grade in this scoped class.
Confining and string-breaking branches are assumption-explicit program statements until
model-specific continuum theorems are closed.

\section{Reproducibility Metadata}

\begin{itemize}
\item date anchor: 2026-02-09 (US),
\item build toolchain tested here: \texttt{/Library/TeX/texbin/pdflatex} (TeX Live 2025),
\item safe build script:
\texttt{\textasciitilde/.codex/skills/pdflatex-safe-build/scripts/build\_pdflatex\_safe.sh}.
\end{itemize}

\section{Conclusion}

Claim 9 should be read as phase-resolved, not universal.
The screened-Abelian branch is fully closed at theorem level in this manuscript,
while non-Abelian confinement and string-breaking crossover remain explicit
\((G,D)\)-indexed next targets.

\end{document}
