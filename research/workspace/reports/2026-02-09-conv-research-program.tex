\documentclass[11pt]{article}
\usepackage[a4paper,margin=1in]{geometry}
\usepackage{amsmath,amssymb,amsthm,mathtools}
\usepackage{bm}
\usepackage{enumitem}
\usepackage{hyperref}

\newtheorem{theorem}{Theorem}[section]
\newtheorem{lemma}[theorem]{Lemma}
\newtheorem{proposition}[theorem]{Proposition}
\newtheorem{corollary}[theorem]{Corollary}
\newtheorem{definition}[theorem]{Definition}
\newtheorem{remark}[theorem]{Remark}

\newcommand{\R}{\mathbb{R}}
\newcommand{\C}{\mathbb{C}}
\newcommand{\eps}{\varepsilon}
\newcommand{\e}{\mathrm{e}}
\newcommand{\Cyl}{\mathrm{Cyl}}
\newcommand{\Cov}{\mathrm{Cov}}
\newcommand{\path}[1]{\texttt{#1}}

\newcommand{\TODO}[1]{\par\noindent\textbf{TODO:} #1\par}

\title{From \path{conv\_patched.md} to a Research Program:\\
Foundational Goals, Proof Targets, Work Done, and Next Steps}
\author{}
\date{2026-02-09 (CET)}

\begin{document}
\maketitle

\begin{abstract}
\textbf{Abstract (scheme).}
\begin{itemize}[leftmargin=1.4em]
\item \path{conv\_patched.md} proposed a foundational slogan:
  representing ``classical solutions'' as Dirac-type localization on extrema,
  and demanding compatibility with refinement/composition, naturally produces
  complex exponential weights and a ``halved'' amplitude whose modulus-square
  yields the concentrating distribution (half-density viewpoint).
\item We turned that conversation into a concrete research program:
  a three-level escalation (statics $\to$ mechanics-in-time $\to$ field theory),
  with explicit dimension gates at the field level (\(d=2\) first, \(d=3\) next,
  \(d=4\) frontier).
\item We then executed a long audit and proved a large scoped subclass of Claim 1:
  exact projective cylinder stability and nontrivial interacting de-regularization
  in a projective oscillatory class, plus a reusable Lean-checked spine of
  algebraic/analytic lemmas supporting small-parameter continuity bounds.
\item What remains open is the global field-level continuum existence bridge in
  genuinely local \(d=3\) and \(d=4\) interacting models, with explicit proof
  obligations stated and partially mechanized sub-lemmas in place.
\end{itemize}
\end{abstract}

\tableofcontents

\section{Reading Instructions (for Humans and Future Agents)}
\label{sec:reading}

\subsection{Source of Truth and Index Files}
\begin{itemize}[leftmargin=1.4em]
\item Primary source: \path{conv\_patched.md} and its rendered companion \path{conv\_patched.pdf}.
\item Research compass: \path{research/workspace/notes/2026-02-09-core-goal-compass.md}.
\item Top-10 claim audit: \path{research/workspace/notes/audits/2026-02-08-top10-claim-audit.md}.
\item Claim 1 scoped proof report: \path{research/workspace/reports/2026-02-09-claim1-scoped-complete-proof.tex}.
\item Lean proof workspace: \path{research/workspace/proofs/} (build via \path{/Users/arivero/.elan/bin/lake}).
\end{itemize}

\subsection{How to Use This Report (Scheme)}
\begin{enumerate}[label=(\arabic*),leftmargin=1.8em]
\item Read \S\ref{sec:conv} to recover what the conversation asserted and why it matters.
\item Read \S\ref{sec:aim}--\S\ref{sec:targets} to see the intended theorem targets and the
  ``success criteria'' for promoting a statement from heuristic to proved.
\item Read \S\ref{sec:done}--\S\ref{sec:status} to see what is already closed and what is still open.
\item Read \S\ref{sec:roadmap} to see the next queue items and why they connect to Goal~1.
\end{enumerate}

\section{The Source Conversation: What Was Known and What Was Claimed}
\label{sec:conv}

\subsection{Conversation Inventory}
\begin{itemize}[leftmargin=1.4em]
\item \textbf{Input target:} transcribe and formalize the paper \path{9803035.pdf} (cited in the conversation)
  ``first for the static case, then with quantum mechanics (action on a time line), then for field theory in \(D=4\),''
  and connect the ``halved expression'' to the literature and to tangent-groupoid ideas.
\item \textbf{Geometry-of-force:} central-force orbit taxonomy across:
  special relativity (SR) power-law forces, Schwarzschild (and higher-$D$ generalizations),
  and gauge potential long-range phases.
\item \textbf{Variational-distribution core:} ``delta of first variation'' as
  \(\delta(f'(x))\) in statics and \(\delta(\delta S/\delta\phi)\) in mechanics/fields.
  Emphasis: this is \emph{not} the derivative distribution \(\delta'\), but a pullback/composition by the map \(f'\).
\item \textbf{Halved/half-density viewpoint:} a square-root object (amplitude)
  is more compositional than a density; modulus-square yields the concentrating density.
  Groupoid convolution is naturally formulated for half-densities.
\item \textbf{Scale/refinement control:} the limit \(\varepsilon\to0\) is only sensible
  if the refinement/composition law is compatible across scales.
  This motivated:
  (i) explicit scale flows (\(\tau_\mu\)),
  (ii) Schwinger--Dyson (integration-by-parts) identities as invariant structure,
  (iii) de-regularization \(\eta\to0^+\) to pass from damped oscillatory weights to purely oscillatory ones.
\end{itemize}

\subsection{Canonical Anchors in \path{conv\_patched.md}}
\textbf{Static variational/distribution seed.}
\begin{itemize}[leftmargin=1.4em]
\item \(\delta(f'(x))\) is identified as the ``obvious'' answer (lines \(\sim 688\)).
\item The halved amplitude is introduced (lines \(\sim 707\)).
\item Tangent groupoid is proposed as a clean conceptual formulation (line \(\sim 814\)).
\item The nondegenerate decomposition \(\delta(f')=\sum \delta(x-x_i)/|f''(x_i)|\) is stated (lines \(\sim 857\)--\(871\)).
\end{itemize}

\textbf{Near-diagonal scaling and half-density interpretation.}
\begin{itemize}[leftmargin=1.4em]
\item ``Near diagonal only'' scaling and why it matters for \(\varepsilon\to0\) (line \(\sim 928\)).
\item Explicit half-density/operator-kernel viewpoint for groupoid convolution (line \(\sim 967\)).
\end{itemize}

\textbf{Escalation to QM and QFT.}
\begin{itemize}[leftmargin=1.4em]
\item Functional delta analogue \(\delta(\delta S/\delta\phi)\) in time-sliced mechanics (lines \(\sim 1149\)--\(1157\)).
\item ``Halved expression becomes propagator amplitude'' viewpoint (lines \(\sim 1122\)--\(1124\)).
\item Synthesis slogan: quantization as refinement-compatible localization (line \(\sim 1209\)).
\end{itemize}

\section{Research Aims (North Star)}
\label{sec:aim}

\subsection{Goal 1 as a Three-Level Program}
\textbf{Primary articulation.}
Goal~1 was locked in \path{research/workspace/notes/theorems/2026-02-09-claim1-three-level-program.md}:
\begin{enumerate}[label=(\arabic*),leftmargin=1.8em]
\item \textbf{Statics (0D action):}
  prototype object \(\delta(f'(x))\), and stationary-phase control of the halved amplitude
  \[
  A_\varepsilon(O)=\varepsilon^{-1/2}\int_{\R} \e^{\frac{i}{\varepsilon}f(x)}O(x)\,dx,
  \qquad
  |A_\varepsilon(O)|^2\to 2\pi\langle\delta(f'),|O|^2\rangle
  \]
  under nondegenerate hypotheses.
\item \textbf{Dynamics (0+1D action):}
  action on time histories \(S[q]=\int L(q,\dot q,t)\,dt\) and variational localization target
  \(\delta(\delta S/\delta q)\), made exact at finite time-slicing as \(\delta(\nabla S_N)\).
\item \textbf{Fields (\(d\)-D action):}
  action \(S[\Phi]=\int \mathcal{L}(\Phi,\partial\Phi,\dots)\,d^dx\),
  formal localization \(\delta(\delta S/\delta \Phi)\),
  and a dimension-gated existence program for regulator removal and reconstruction.
\end{enumerate}

\subsection{Dimension-Gated Field Existence Ladder}
\textbf{Roadmap lock.}
Field-level work is dimension-indexed by design:
\begin{itemize}[leftmargin=1.4em]
\item \(d=2\): strongest closure candidate; we closed an interacting ultralocal \(\phi^4\) field-indexed theorem (AP).
\item \(d=3\): intermediate bridge rung; candidate theorem stated with explicit proof obligations B1--B5 (AS).
\item \(d=4\): physically central frontier; explicit lift-obstruction sheet enumerates missing inputs (AQ).
\end{itemize}

\textbf{Existence notions.}
At the field level, we separate:
\begin{enumerate}[label=(\roman*),leftmargin=1.8em]
\item regulated existence (finite cutoff objects),
\item continuum existence (regulator removal with nontrivial limit),
\item physical reconstruction (Euclidean-to-Minkowski, or equivalent).
\end{enumerate}

\section{What We Expect to Prove (Proof Targets and Success Criteria)}
\label{sec:targets}

\subsection{Claim 1 (Foundational Claim)}
\textbf{Core slogan (scheme).}
Claim~1 is the bridge:
\[
\text{``halved'' amplitude / half-density}
\quad\leadsto\quad
\text{density-level localization on extrema}
\quad\leadsto\quad
\delta\!\left(\frac{\delta S}{\delta \phi}\right)\ \text{(QM/QFT)}.
\]

\textbf{Scoped proof target.}
In this repository, Claim~1 is treated as a hierarchy:
\begin{enumerate}[label=(\roman*),leftmargin=1.8em]
\item \emph{finite-dimensional theorem-grade}: explicit stationary phase / distributional localization;
\item \emph{regulated infinite family theorem-grade}: projective cylinder stability, SD closure, and de-regularization in a specified class;
\item \emph{field-level theorem-grade}: dimension-indexed continuum existence with SD pass-through and \(c\)-invariance, plus reconstruction where applicable.
\end{enumerate}

\textbf{Current status label.}
The top-10 audit keeps Claim~1 labelled \texttt{heuristic}, while recording that a large scoped core is proved
(\path{research/workspace/reports/2026-02-09-claim1-scoped-complete-proof.tex}).

\subsection{Auxiliary Claims (2--10)}
\textbf{Role.}
Claims 2--10 are the ``geometry-of-force'' and model-taxonomy support.
They ensure the action-reduction mechanism is explicit and correct in concrete SR/GR/gauge settings.

\textbf{Current theorem-grade closure (scheme).}
Per the top-10 audit, the following are \texttt{proved} as theorem notes:
\begin{itemize}[leftmargin=1.4em]
\item SR center-access trichotomy (Claim~2) and SR Coulomb phase portrait (Claim~3),
\item Duffing-type \(n=3\) reduction (Claim~4),
\item \(D\)-dimensional GR matching (Claim~5),
\item Schwarzschild bound-orbit interval and ISCO framing (Claims~6--7),
\item selected higher-$D$ GR and gauge taxonomy branches (Claims~8--9) are partially closed; remaining global branches are tracked as open,
\item SR circular-orbit benchmark inequalities (Claim~10) are formalized as regression identities.
\end{itemize}

\subsection{What Counts as a Proof Here}
\textbf{Validation contract (scheme).}
Every upgraded statement is expected to include:
\begin{enumerate}[label=(\arabic*),leftmargin=1.8em]
\item explicit assumptions (model, regime, approximations),
\item units/dimensions check (where relevant),
\item symmetry/conservation checks (where relevant),
\item at least one independent cross-check path:
  Lean proof or symbolic derivation or numerical diagnostics,
\item a confidence/uncertainty note,
\item reproducibility metadata (commands, versions).
\end{enumerate}
\textbf{Rule of thumb.} Half-density/groupoid composition is \emph{kinematic};
continuum-limit existence is \emph{dynamical} and requires separate analytic gates.

\section{What Has Been Done So Far (Artifacts and Results)}
\label{sec:done}

\subsection{Audit Results}
\textbf{Top-10 audit lock.}
The canonical audit is \path{research/workspace/notes/audits/2026-02-08-top10-claim-audit.md}.

\textbf{Key audit outcomes (scheme).}
\begin{itemize}[leftmargin=1.4em]
\item Claim~1 remains the highest-novelty, highest-risk item, but has a nontrivial scoped proof closure report.
\item Claims~2--7 and Claim~10 are \texttt{proved} in dedicated theorem notes; Claims~8--9 are partially closed with explicit open sectors.
\item The audit enforces a non-drift filter: new work must strengthen
  (i) variational-distribution core, (ii) geometry-of-force bridge, or (iii) scale-control core.
\end{itemize}

\textbf{Current ``next'' audit item (scheme).}
As of the latest update, the next queued Lean target is AN-18:
reduce finite exponential-family increment hypotheses by proving automatic regularity and non-vanishing conditions.

\subsection{Theorem Notes and Reports}
\textbf{Core reports (scheme).}
\begin{itemize}[leftmargin=1.4em]
\item \path{research/workspace/reports/2026-02-08-claim1-variational-delta-note.tex}:
  static $\to$ QM $\to$ QFT ladder and \(\delta(\partial S)\) viewpoint.
\item \path{research/workspace/reports/2026-02-09-claim1-scoped-complete-proof.tex}:
  scoped complete proof in a projective oscillatory class (projective stability, counterterms, de-regularization, SD, \(\tau_\mu\)).
\item \path{research/workspace/reports/2026-02-09-newton-action-path-integral-lecture.md}:
  Newton $\to$ action reduction $\to$ oscillatory localization lecture narrative.
\end{itemize}

\textbf{Field-level dimension notes (scheme).}
\begin{itemize}[leftmargin=1.4em]
\item \path{research/workspace/notes/theorems/2026-02-09-claim1-d2-ultralocal-phi4-closure.md}:
  theorem-grade field-indexed closure in \(d=2\) ultralocal interacting class.
\item \path{research/workspace/notes/theorems/2026-02-09-claim1-d3-intermediate-bridge-candidate.md}:
  \(d=3\) bridge candidate with proof obligations B1--B5.
\item \path{research/workspace/notes/theorems/2026-02-09-claim1-d4-lift-obstruction-sheet.md}:
  explicit failure points for naive \(d=2\to d=4\) lift in local interacting models.
\end{itemize}

\subsection{Lean Formalization (Machine-Checked Spine)}
\textbf{Lean workspace.}
The Lean project lives in \path{research/workspace/proofs/}.

\textbf{Machine-checked modules (scheme).}
As recorded in \path{research/workspace/notes/theorems/2026-02-09-claim1-lean-formalization-status.md},
the current Lean spine includes:
\begin{itemize}[leftmargin=1.4em]
\item \(c\)-invariance under \(\tau_\mu\) scaling and composition laws,
\item quotient-derivative covariance form for \(\omega=N/Z\),
\item finite-average covariance inequality templates,
\item derivative-bound and interval-increment templates for ratio states,
\item finite exponential-family bridges:
  derivative-under-sum, centered representation, derivative bounds, and a model-internal \(C\kappa\) increment bound.
\end{itemize}

\subsection{Simulation and Symbolic Checks}
\textbf{Scope.}
Python scripts are used as \emph{diagnostics}, not as proof substitutes.
The index is kept in \path{research/workspace/notes/README.md}.

\textbf{Examples (scheme).}
\begin{itemize}[leftmargin=1.4em]
\item \(d=2\) ultralocal closure diagnostic:
  \path{research/workspace/simulations/claim1_d2_ultralocal_phi4_closure_check.py}.
\item half-density kinematic vs dynamical counterexample diagnostic:
  \path{research/workspace/simulations/claim1_halfdensity_kinematic_dynamic_split_check.py}.
\item \(d=3\) bridge toy scan:
  \path{research/workspace/simulations/claim1_d3_bridge_toy_coupling_scan.py}.
\end{itemize}

\section{Current Status of Claim 1 (Closed Pieces, Open Gaps)}
\label{sec:status}

\subsection{The Scoped Complete Proof and Its Scope}
\textbf{Scoped closure (scheme).}
\path{research/workspace/reports/2026-02-09-claim1-scoped-complete-proof.tex} proves, in a projective cylinder class:
\begin{itemize}[leftmargin=1.4em]
\item exact projective stability on cylinder observables,
\item a well-defined continuum state on cylinder observables,
\item constructive counterterm repair mechanisms,
\item de-regularization \(\eta\to0^+\) in several interacting families,
\item Schwinger--Dyson identities and exact \(\tau_\mu\)-type scale-flow covariance,
\item explicit large-\(N\) lifts and non-factorized interacting extensions under stated conditions.
\end{itemize}

\textbf{What it does not claim (scheme).}
\begin{itemize}[leftmargin=1.4em]
\item it does not claim a full continuum local QFT construction in \(d=4\),
\item it does not identify a unique interacting continuum field theory beyond the cylinder/regulated scope,
\item it does not bypass the dimension-gated existence and reconstruction obligations.
\end{itemize}

\subsection{Statics and Dynamics: Where the Delta-of-Variation Logic Is Theorem-Grade}
\textbf{Static level (scheme).}
\begin{itemize}[leftmargin=1.4em]
\item In finite dimensions, stationary phase yields \(|A_\varepsilon|^2\to 2\pi\langle \delta(f'),|O|^2\rangle\)
  in nondegenerate regimes.
\item The distributional identity \(\delta(f')=\sum \delta_{x_i}/|f''(x_i)|\) reduces the limiting object
  to a finite measure supported on critical points (amenable to formal proof assistants).
\end{itemize}

\textbf{Dynamics level (scheme).}
\begin{itemize}[leftmargin=1.4em]
\item time-slicing makes \(\delta(\delta S/\delta q)\) exact as \(\delta(\nabla S_N)\) at fixed discretization,
\item continuum claims require limit control under refinement, tracked by projective/cylinder programs and scale-flow constraints.
\end{itemize}

\subsection{Field Level: \(d=2\) Closure, \(d=3\) Bridge, \(d=4\) Obstruction}
\textbf{\(d=2\) closure (scheme).}
\begin{itemize}[leftmargin=1.4em]
\item \(d=2\) ultralocal interacting \(\phi^4\) model: exact cylinder stabilization and continuum existence,
  field-level SD identity, and exact \(c\)-invariance along \(\tau_\mu\)-orbits.
\item interpretation: this is non-Gaussian but ultralocal, serving as a controlled baseline rather than full local QFT.
\end{itemize}

\textbf{\(d=3\) bridge candidate (scheme).}
\begin{itemize}[leftmargin=1.4em]
\item add nearest-neighbor gradient coupling \(\kappa\ge 0\) and seek a small-\(\kappa\) regime \([0,\kappa_*]\),
  with explicit proof obligations:
  B1 moments, B2 tightness, B3 non-vanishing denominator, B4 SD insertion control, B5 \(\kappa\)-continuity.
\item Lean work mechanizes B5-shaped increment bounds in abstract and in a finite exponential-family toy class;
  the field-level model bridge remains open.
\end{itemize}

\textbf{\(d=4\) obstruction sheet (scheme).}
\begin{itemize}[leftmargin=1.4em]
\item restoring local propagation destroys product structure and exact cylinder decoupling;
  SD hierarchy is no longer one-site closed.
\item regulator removal requires explicit renormalization flow control and nontriviality criteria.
\item half-density statements must be split into kinematic vs dynamical (AR note).
\end{itemize}

\section{How We Continue (Roadmap)}
\label{sec:roadmap}

\subsection{Immediate Next Targets}
\textbf{Immediate queue (scheme).}
\begin{itemize}[leftmargin=1.4em]
\item AN-18 (Lean): reduce finite exponential-family increment hypotheses by proving automatic regularity and non-vanishing conditions.
\item Field-level: close at least one genuine beyond-ultralocal \(d=3\) statement by addressing B1--B5 systematically.
\item Cross-level: keep half-density composition results separate from continuum existence, and ensure every ``lift'' names its missing analytic gates.
\end{itemize}

\subsection{Longer-Horizon Conjectures and Risks}
\textbf{Core missing analytic ingredients (scheme).}
\begin{itemize}[leftmargin=1.4em]
\item tightness/precompactness mechanisms for cylinder marginals in local interacting field models,
\item robust non-vanishing control for oscillatory partition normalizations on complex parameter domains,
\item SD pass-through with gradient terms (control of insertions and boundary terms),
\item \(d=4\) renormalization flow control and nontriviality criteria,
\item reconstruction to Minkowski (when pursuing that branch).
\end{itemize}

\section{Appendices}
\label{sec:appendix}

\subsection{Glossary}
\textbf{Core glossary (scheme).}
\begin{itemize}[leftmargin=1.4em]
\item \(c\)-parameter:
  \(c := (\eta - i/h)\kappa\).
\item \(c\)-invariant:
  invariant under parameter changes that keep \(c\) fixed; constant on \(\tau_\mu\)-orbits.
\item \(\tau_\mu\) flow:
  \(\tau_\mu:(\kappa,\eta,h)\mapsto(\mu\kappa,\eta/\mu,\mu h)\), \(\mu>0\).
\item de-regularization:
  one-sided limit \(\eta\to0^+\) from damped oscillatory weights to purely oscillatory ones.
\item SD identity (scoped finite form):
  \(\langle \partial_i\psi\rangle_c = c\,\langle \psi\,\partial_i S\rangle_c\).
\item half-density (here):
  amplitude-level object whose modulus-square yields density-level quantities; natural for groupoid convolution kernels.
\end{itemize}

\subsection{Reproducibility (Commands and Entry Points)}
\textbf{Lean build (scheme).}
\begin{verbatim}
cd research/workspace/proofs
/Users/arivero/.elan/bin/lake update
/Users/arivero/.elan/bin/lake build Claim1lean
\end{verbatim}

\textbf{Selected diagnostics (scheme).}
\begin{verbatim}
python3.12 research/workspace/simulations/claim1_d2_ultralocal_phi4_closure_check.py
python3.12 research/workspace/simulations/claim1_halfdensity_kinematic_dynamic_split_check.py
\end{verbatim}

\subsection{File Map}
\textbf{Curated index (scheme).}
\begin{itemize}[leftmargin=1.4em]
\item source: \path{conv\_patched.md}, \path{conv\_patched.pdf}.
\item audit: \path{research/workspace/notes/audits/2026-02-08-top10-claim-audit.md}.
\item compass: \path{research/workspace/notes/2026-02-09-core-goal-compass.md}.
\item Claim~1 reports: \path{research/workspace/reports/2026-02-08-claim1-variational-delta-note.tex},
  \path{research/workspace/reports/2026-02-09-claim1-scoped-complete-proof.tex}.
\item synthesis report: \path{research/workspace/reports/2026-02-09-newton-action-path-integral-lecture.md}.
\item Lean: \path{research/workspace/proofs/Claim1lean.lean} and \path{research/workspace/proofs/Claim1lean/}.
\item theorem notes index: \path{research/workspace/notes/README.md}.
\end{itemize}

\end{document}
