\documentclass[11pt]{article}
\usepackage[a4paper,margin=1in]{geometry}
\usepackage{amsmath,amssymb,amsthm,mathtools}
\usepackage{bm}
\usepackage{enumitem}
\usepackage{hyperref}

\newtheorem{theorem}{Theorem}[section]
\newtheorem{lemma}[theorem]{Lemma}
\newtheorem{proposition}[theorem]{Proposition}
\newtheorem{corollary}[theorem]{Corollary}
\newtheorem{definition}[theorem]{Definition}
\newtheorem{remark}[theorem]{Remark}

\newcommand{\R}{\mathbb{R}}
\newcommand{\C}{\mathbb{C}}
\newcommand{\eps}{\varepsilon}
\newcommand{\e}{\mathrm{e}}
\newcommand{\Cyl}{\mathrm{Cyl}}
\newcommand{\Cov}{\mathrm{Cov}}
\newcommand{\path}[1]{\texttt{#1}}

\newcommand{\TODO}[1]{\par\noindent\textbf{TODO:} #1\par}

\title{From \path{conv\_patched.md} to a Research Program:\\
Foundational Goals, Proof Targets, Work Done, and Next Steps}
\author{}
\date{2026-02-09 (CET)}

\begin{document}
\maketitle

\begin{abstract}
This report is the canonical ``what we are doing'' document for the
\path{conv\_patched.md} conversation and the research produced afterwards in this repository.
The source conversation proposed a foundational slogan:
representing ``classical solutions'' as Dirac-type localization on extrema, and then insisting
that this localization be compatible with refinement and composition, naturally generates
complex exponential weights \(\e^{\frac{i}{\hbar}S}\) and a ``halved'' object that behaves like an
amplitude (or half-density) whose modulus-square yields the concentrating distribution.

We turned that slogan into a concrete, dimension-aware research program:
a three-level escalation (statics $\to$ mechanics-in-time $\to$ field theory), with half-density
as an \emph{optional} formalism at each level, and with explicit gates for field-level existence
by spacetime dimension (\(d=2\) first, \(d=3\) next, \(d=4\) frontier).

On the theorem side, we proved a large scoped subclass of Claim~1:
exact projective stability for cylinder observables, constructive counterterm repair, rigorous
Schwinger--Dyson and scale-flow (\(\tau_\mu\)) structure, and nontrivial de-regularization
\(\eta\to0^+\) in a projective oscillatory class.
On the formal side, we built a Lean-checked spine of reusable lemmas (quotient-derivative covariance
form, centered bounds, and small-parameter increment control) and specialized that spine to a
concrete finite exponential-family model.

What remains open is the genuinely local field-level bridge in interacting models beyond ultralocality:
in \(d=3\), a precise candidate theorem is stated with explicit obligations; in \(d=4\), we have an
explicit obstruction sheet explaining why naive lifts fail without renormalization and nontriviality input.
\end{abstract}

\tableofcontents

\section{Reading Instructions (for Humans and Future Agents)}
\label{sec:reading}

\subsection{Source of Truth and Index Files}
The primary source is \path{conv\_patched.md}, with a rendered companion \path{conv\_patched.pdf}.
All claims, program decisions, and anchors in this report should be traced back either to that source
conversation or to one of the theorem-grade artifacts produced in this workspace.

The minimal entry points are:
\begin{itemize}[leftmargin=1.4em]
\item \path{research/workspace/notes/2026-02-09-core-goal-compass.md} (program compass and ``north star''),
\item \path{research/workspace/notes/audits/2026-02-08-top10-claim-audit.md} (current audit labels and the next queue item),
\item \path{research/workspace/reports/2026-02-09-claim1-scoped-complete-proof.tex} (the main scoped Claim~1 proof report),
\item \path{research/workspace/proofs/} (Lean formalization workspace, built with \path{/Users/arivero/.elan/bin/lake}).
\end{itemize}

\subsection{How to Use This Report}
This report is meant to support two readers:
humans who want a narrative of what is known and what is not, and future agents/models that need a
compact, unambiguous summary of the proof targets and the current state of the work tree.
The recommended path is:
\begin{enumerate}[label=(\arabic*),leftmargin=1.8em]
\item Read \S\ref{sec:conv} to recover the \emph{problem statement} as it appeared in the conversation.
\item Read \S\ref{sec:aim}--\S\ref{sec:targets} to see the \emph{research program} and the proof standards used here.
\item Read \S\ref{sec:done}--\S\ref{sec:status} to see the \emph{executed results} (what is theorem-grade today).
\item Read \S\ref{sec:roadmap} to see the \emph{next targets} and the explicit gaps that motivate them.
\end{enumerate}

\section{The Source Conversation: What Was Known and What Was Claimed}
\label{sec:conv}

\subsection{Conversation Inventory}
The conversation was not a generic discussion of ``AI in physics.'' It was an explicit technical request:
take an existing mathematical manuscript (referred to in the conversation as \path{9803035.pdf}) and treat it as
a source for a foundational research program. The requested escalation was concrete and ordered:
first the static case (finite-dimensional oscillatory integrals), then quantum mechanics (action integrated in a time
coordinate, with time slicing), then field theory (action integrated in spacetime, with particular interest in
\(D=4\)), while keeping track of the ``halved expression'' and its interpretation in the literature and in geometric
frameworks (tangent groupoids and renormalization-style refinement control).

At the same time, the conversation contained a second technical axis: a large body of central-force dynamics in SR/GR
and gauge settings (power-law forces, Coulomb, Schwarzschild, higher-dimensional black holes, and long-range phase
taxonomies). This axis matters because it makes the ``action reduction'' mechanism explicit and testable in many
examples, and because it provides a concrete target for the slogan ``geometry-of-force emerges from variational
mechanics''.

Finally, the conversation centered on a specific distributional object and a specific compositional intuition.
The distributional object is the ``delta of first variation'':
in statics, \(\delta(f'(x))\); in mechanics and fields, the functional delta
\(\delta(\delta S/\delta \phi)\) or \(\delta(\delta S/\delta \Phi)\).
This is emphatically \emph{not} the derivative distribution \(\delta'\); it is a pullback/composition of the Dirac
delta by the map \(f'\) (or the Euler--Lagrange map). The compositional intuition is that amplitudes are the correct
objects to compose, not probabilities: the ``halved expression'' behaves like an amplitude (or half-density), and
modulus-square produces a density-level localizing object.

The conversation also made an explicit control demand: the relevant limits (e.g.\ \(\varepsilon\to0\) in oscillatory
integrals, or refinement limits in time slicing and lattice field theory) must be compatible with composition across
scales. This is where \(\tau_\mu\)-type scale flows, Schwinger--Dyson identities, and de-regularization \(\eta\to0^+\)
enter the program.

\subsection{Canonical Anchors in \path{conv\_patched.md}}
The anchors below are the minimal set of \path{conv\_patched.md} lines that define the program.
Line numbers refer to the \emph{patched} transcript, so they should be treated as stable for this repository.

\paragraph{Static seed.}
The conversation identifies \(\delta(f'(x))\) as the relevant concentrating object (\path{conv\_patched.md:688}),
introduces the ``halved'' amplitude (\path{conv\_patched.md:707}), and records the standard nondegenerate identity
\[
\delta(f'(x))=\sum_{x_i\in\mathrm{Crit}(f)}\frac{\delta(x-x_i)}{|f''(x_i)|}
\]
(\path{conv\_patched.md:871}).
These items are the static prototype of ``delta of first variation''.

\paragraph{Near-diagonal scaling and half-densities.}
The conversation emphasizes that the relevant \(\varepsilon\to0\) scaling is near-diagonal (\(y\approx x\)),
not a far-away variable limit (\path{conv\_patched.md:928}). It also explicitly states the geometric reading:
on a Lie groupoid, convolution is naturally formulated using half-densities so that composition is coordinate-free
(\path{conv\_patched.md:967}). This is the conceptual reason the ``halved expression'' is treated as an amplitude-like
object in the program.

\paragraph{Escalation to mechanics and fields.}
For quantum mechanics, the halved object is identified with the propagator amplitude
(\path{conv\_patched.md:1122}) and the static \(\delta(f')\) is upgraded to the functional delta
\(\delta(\delta S/\delta\phi)\) (\path{conv\_patched.md:1149}).
The synthesis line is that quantization can be read as the refinement-compatible version of classical localization
(\path{conv\_patched.md:1209}): the same localization object appears at each level, but compatibility with composition
across scales forces nontrivial structure (a surviving finite deformation parameter).

\section{Research Aims (North Star)}
\label{sec:aim}

\subsection{Goal 1 as a Three-Level Program}
Goal~1 is the foundational target of this repository. It is intentionally framed as a three-level escalation
because the same conceptual object changes character as the ambient dimension changes:
in statics we can literally write down distributions like \(\delta(f')\); in mechanics we can discretize time
to obtain honest finite-dimensional deltas \(\delta(\nabla S_N)\); in fields we have to confront a genuinely
infinite-dimensional limit problem with dimension-dependent existence and renormalization behavior.

The program is pinned to \path{research/workspace/notes/theorems/2026-02-09-claim1-three-level-program.md}.
The level separation is not rhetorical; it is the main mechanism for preventing over-claims.
\emph{Half-density language is optional at every level:} it is structurally useful for composition statements,
but it is not accepted as a substitute for the dynamical work needed to pass to continuum limits.

The three levels are:
\begin{enumerate}[label=(\arabic*),leftmargin=1.8em]
\item \textbf{Statics (0D action):}
  prototype object \(\delta(f'(x))\), and stationary-phase control of the halved amplitude
  \[
  A_\varepsilon(O)=\varepsilon^{-1/2}\int_{\R} \e^{\frac{i}{\varepsilon}f(x)}O(x)\,dx,
  \qquad
  |A_\varepsilon(O)|^2\to 2\pi\langle\delta(f'),|O|^2\rangle
  \]
  under nondegenerate hypotheses.
\item \textbf{Dynamics (0+1D action):}
  action on time histories \(S[q]=\int L(q,\dot q,t)\,dt\) and variational localization target
  \(\delta(\delta S/\delta q)\), made exact at finite time-slicing as \(\delta(\nabla S_N)\).
\item \textbf{Fields (\(d\)-D action):}
  action \(S[\Phi]=\int \mathcal{L}(\Phi,\partial\Phi,\dots)\,d^dx\),
  formal localization \(\delta(\delta S/\delta \Phi)\),
  and a dimension-gated existence program for regulator removal and reconstruction.
\end{enumerate}

\subsection{Dimension-Gated Field Existence Ladder}
The field-level part of the program is explicitly dimension-gated. This is not just ``because physics is \(3+1\)'',
but because the analytic existence theory is dimension-sensitive: the same local interaction can be benign in one
dimension and critical or ill-behaved in another. The repository therefore treats \(d\) as a first-class index in
every field-level statement, and it separates three notions of ``existence'' (regulated, continuum, reconstruction)
before any claim is allowed to upgrade to ``field-level closure''.

The roadmap is pinned to \path{research/workspace/notes/theorems/2026-02-09-claim1-field-dimension-existence-roadmap.md}.
In that roadmap, field-level work is dimension-indexed by design:
\begin{itemize}[leftmargin=1.4em]
\item \(d=2\): strongest closure candidate; we closed an interacting ultralocal \(\phi^4\) field-indexed theorem (AP).
\item \(d=3\): intermediate bridge rung; candidate theorem stated with explicit proof obligations B1--B5 (AS).
\item \(d=4\): physically central frontier; explicit lift-obstruction sheet enumerates missing inputs (AQ).
\end{itemize}

At the field level, we separate:
\begin{enumerate}[label=(\roman*),leftmargin=1.8em]
\item regulated existence (finite cutoff objects),
\item continuum existence (regulator removal with nontrivial limit),
\item physical reconstruction (Euclidean-to-Minkowski, or equivalent).
\end{enumerate}

The main pragmatic consequence is that \(d=2\) results are treated as a closure target for the \emph{logic} of Claim~1
in an interacting class, \(d=3\) results are treated as the next bridge rung where local propagation is reintroduced,
and \(d=4\) results are treated as a frontier where one must explicitly account for renormalization and nontriviality.

\section{What We Expect to Prove (Proof Targets and Success Criteria)}
\label{sec:targets}

\subsection{Claim 1 (Foundational Claim)}
Claim~1 is the foundational bridge that motivated the entire work tree.
It starts from a static distributional fact (\(\delta(f')\) localizes on extrema) and escalates it to mechanics
and fields by replacing \(f\) with an action functional \(S\).
The slogan can be stated compactly as:
\[
\text{``halved'' amplitude / half-density}
\quad\leadsto\quad
\text{density-level localization on extrema}
\quad\leadsto\quad
\delta\!\left(\frac{\delta S}{\delta \phi}\right)\ \text{(QM/QFT)}.
\]

The ``halved'' object is important because it is the compositional object: it behaves like an amplitude or
half-density under convolution/composition (this is the groupoid-kernel intuition), while the squaring step produces
the density-level object that concentrates on classical solutions.
In finite dimensions this is standard stationary phase; the research challenge is to make the limit/refinement
procedures honest across the time and field escalations.

To prevent ambiguity, this repository treats Claim~1 as a hierarchy of increasingly strong targets:
\begin{enumerate}[label=(\roman*),leftmargin=1.8em]
\item \emph{Finite-dimensional theorem-grade:} explicit stationary phase / distributional localization statements.
\item \emph{Regulated infinite family theorem-grade:} a specified projective/cylinder family of normalized oscillatory
  states with exact stability, explicit control of renormalization/counterterms, SD closure, and de-regularization
  \(\eta\to0^+\) in that class.
\item \emph{Field-level theorem-grade:} dimension-indexed continuum existence with SD pass-through and \(c\)-invariance,
  and, when pursuing that branch, a reconstruction step connecting Euclidean correlators to a Minkowski theory.
\end{enumerate}

The audit keeps Claim~1 labelled \texttt{heuristic} at the global level, because the full bridge to a unique local
interacting continuum field theory is not yet proved. However, a large scoped core is now theorem-grade and recorded
in \path{research/workspace/reports/2026-02-09-claim1-scoped-complete-proof.tex}.

\subsection{Auxiliary Claims (2--10)}
Claims 2--10 are the ``geometry-of-force'' and model-taxonomy support.
They ensure the action-reduction mechanism is explicit and correct in concrete SR/GR/gauge settings.

They also play a practical role: they provide regression targets and sanity checks.
If the foundational track produces a formalism that contradicts known orbit classifications (e.g.\ SR Coulomb
precession regimes or Schwarzschild separatrices), that is immediate evidence of a mis-specified definition or
an incorrect scaling assumption.

Per the top-10 audit, the following are \texttt{proved} as theorem notes (with explicit derivations and cross-checks):
\begin{itemize}[leftmargin=1.4em]
\item SR center-access trichotomy (Claim~2) and SR Coulomb phase portrait (Claim~3),
\item Duffing-type \(n=3\) reduction (Claim~4),
\item \(D\)-dimensional GR matching (Claim~5),
\item Schwarzschild bound-orbit interval and ISCO framing (Claims~6--7),
\item selected higher-$D$ GR and gauge taxonomy branches (Claims~8--9) are partially closed; remaining global branches are tracked as open,
\item SR circular-orbit benchmark inequalities (Claim~10) are formalized as regression identities.
\end{itemize}

\subsection{What Counts as a Proof Here}
This repository uses an explicit validation contract. The intention is to prevent subtle drift:
no statement is upgraded to ``proved'' unless it states its assumptions and passes at least one independent
verification route (formal proof when feasible; otherwise symbolic or numerical checks).

Every upgraded statement is expected to include:
\begin{enumerate}[label=(\arabic*),leftmargin=1.8em]
\item explicit assumptions (model, regime, approximations),
\item units/dimensions check (where relevant),
\item symmetry/conservation checks (where relevant),
\item at least one independent cross-check path:
  Lean proof or symbolic derivation or numerical diagnostics,
\item a confidence/uncertainty note,
\item reproducibility metadata (commands, versions).
\end{enumerate}

The key rule of thumb is the kinematic/dynamical split:
half-density and groupoid convolution statements certify \emph{kinematic} consistency of composition laws at fixed
regulator, but they do not certify \emph{dynamical} convergence as the regulator is removed.
Continuum-limit claims require separate analytic gates (tightness, non-vanishing normalization, and closure of SD
identities in the limit). This split is spelled out in
\path{research/workspace/notes/theorems/2026-02-09-claim1-halfdensity-kinematic-dynamic-split.md}.

\section{What Has Been Done So Far (Artifacts and Results)}
\label{sec:done}

\subsection{Audit Results}
The canonical audit is \path{research/workspace/notes/audits/2026-02-08-top10-claim-audit.md}.
Its job is to keep the work focused and to prevent ``proof by accumulation'' (writing many notes without closing
the foundational gaps). It does that in two ways:
(i) it assigns rigor labels (\texttt{proved}, \texttt{heuristic}, \texttt{speculative}) with explicit upgrade paths,
and (ii) it enforces a non-drift filter so that new work must strengthen one of the three cores:
variational-distribution, geometry-of-force, or scale-control.

The most important audit outcomes so far are:
\begin{itemize}[leftmargin=1.4em]
\item Claim~1 remains the highest-novelty, highest-risk item, but has a nontrivial scoped proof closure report.
\item Claims~2--7 and Claim~10 are \texttt{proved} in dedicated theorem notes; Claims~8--9 are partially closed with explicit open sectors.
\item The audit enforces a non-drift filter: new work must strengthen
  (i) variational-distribution core, (ii) geometry-of-force bridge, or (iii) scale-control core.
\end{itemize}

As of the latest update, the next queued Lean target is AN-18:
reduce the finite exponential-family increment hypotheses by proving automatic regularity and non-vanishing conditions.
This is intentionally modest: it is a ``hygiene'' step that reduces assumptions in the mechanized spine and keeps the
field-level bridge obligations clearly separated from toy-model artifacts.

\subsection{Theorem Notes and Reports}
This repository separates two types of human-readable artifacts:
\emph{theorem notes} (\path{research/workspace/notes/theorems/}) and
\emph{reports} (\path{research/workspace/reports/}).
The theorem notes are bite-sized upgrades of individual claims with explicit assumptions and proofs; the reports are
longer manuscripts that weave multiple notes into a single dependency chain.

The current core reports are:
\begin{itemize}[leftmargin=1.4em]
\item \path{research/workspace/reports/2026-02-08-claim1-variational-delta-note.tex}:
  static $\to$ QM $\to$ QFT ladder and \(\delta(\partial S)\) viewpoint.
\item \path{research/workspace/reports/2026-02-09-claim1-scoped-complete-proof.tex}:
  scoped complete proof in a projective oscillatory class (projective stability, counterterms, de-regularization, SD, \(\tau_\mu\)).
\item \path{research/workspace/reports/2026-02-09-newton-action-path-integral-lecture.md}:
  Newton $\to$ action reduction $\to$ oscillatory localization lecture narrative.
\end{itemize}

At the field level, three notes define the current dimension-gated status:
\begin{itemize}[leftmargin=1.4em]
\item \path{research/workspace/notes/theorems/2026-02-09-claim1-d2-ultralocal-phi4-closure.md}:
  theorem-grade field-indexed closure in \(d=2\) ultralocal interacting class.
\item \path{research/workspace/notes/theorems/2026-02-09-claim1-d3-intermediate-bridge-candidate.md}:
  \(d=3\) bridge candidate with proof obligations B1--B5.
\item \path{research/workspace/notes/theorems/2026-02-09-claim1-d4-lift-obstruction-sheet.md}:
  explicit failure points for naive \(d=2\to d=4\) lift in local interacting models.
\end{itemize}

\subsection{Lean Formalization (Machine-Checked Spine)}
The Lean project lives in \path{research/workspace/proofs/}. Its role is not to formalize the entire analytic
infrastructure of stationary phase or continuum QFT; it is to machine-check a reusable spine of algebraic and
finite-dimensional analytic lemmas that show up repeatedly in the Claim~1 program (especially in small-parameter
continuity bounds and in ratio-state control).

As recorded in \path{research/workspace/notes/theorems/2026-02-09-claim1-lean-formalization-status.md},
the current machine-checked modules include:
\begin{itemize}[leftmargin=1.4em]
\item \(c\)-invariance under \(\tau_\mu\) scaling and composition laws,
\item quotient-derivative covariance form for \(\omega=N/Z\),
\item finite-average covariance inequality templates,
\item derivative-bound and interval-increment templates for ratio states,
\item finite exponential-family bridges:
  derivative-under-sum, centered representation, derivative bounds, and a model-internal \(C\kappa\) increment bound.
\end{itemize}

\subsection{Simulation and Symbolic Checks}
Python scripts are used as \emph{diagnostics}, not as proof substitutes. Their purpose is to provide fast sanity
checks (e.g.\ sign errors, scaling regimes, stability criteria) and to explore toy parameter scans for candidate
theorems. The canonical index is kept in \path{research/workspace/notes/README.md}.

Representative examples include:
\begin{itemize}[leftmargin=1.4em]
\item \(d=2\) ultralocal closure diagnostic:
  \path{research/workspace/simulations/claim1_d2_ultralocal_phi4_closure_check.py}.
\item half-density kinematic vs dynamical counterexample diagnostic:
  \path{research/workspace/simulations/claim1_halfdensity_kinematic_dynamic_split_check.py}.
\item \(d=3\) bridge toy scan:
  \path{research/workspace/simulations/claim1_d3_bridge_toy_coupling_scan.py}.
\end{itemize}

\section{Current Status of Claim 1 (Closed Pieces, Open Gaps)}
\label{sec:status}

\subsection{The Scoped Complete Proof and Its Scope}
The main theorem-grade closure for Claim~1 in this repository is the scoped report
\path{research/workspace/reports/2026-02-09-claim1-scoped-complete-proof.tex}.
It proves Claim~1 in a specified but nontrivial projective/cylinder class of oscillatory states.
At a high level, the closure consists of two parts:
(i) \emph{exact projective stability} for cylinder observables (a discrete analogue of refinement invariance),
and (ii) \emph{analytic control} sufficient to justify de-regularization and the persistence of SD/\(\tau_\mu\)
structure in that class.

Concretely, within the stated block-tail and interacting extensions, the report proves:
\begin{itemize}[leftmargin=1.4em]
\item exact projective stability on cylinder observables,
\item a well-defined continuum state on cylinder observables,
\item constructive counterterm repair mechanisms,
\item de-regularization \(\eta\to0^+\) in several interacting families,
\item Schwinger--Dyson identities and exact \(\tau_\mu\)-type scale-flow covariance,
\item explicit large-\(N\) lifts and non-factorized interacting extensions under stated conditions.
\end{itemize}

It is equally important to record what the scoped report does \emph{not} claim:
\begin{itemize}[leftmargin=1.4em]
\item it does not claim a full continuum local QFT construction in \(d=4\),
\item it does not identify a unique interacting continuum field theory beyond the cylinder/regulated scope,
\item it does not bypass the dimension-gated existence and reconstruction obligations.
\end{itemize}

\subsection{Statics and Dynamics: Where the Delta-of-Variation Logic Is Theorem-Grade}
At the static level, the delta-of-variation logic is theorem-grade under standard nondegeneracy hypotheses.
The core facts are:
\begin{itemize}[leftmargin=1.4em]
\item In finite dimensions, stationary phase yields \(|A_\varepsilon|^2\to 2\pi\langle \delta(f'),|O|^2\rangle\)
  in nondegenerate regimes.
\item The distributional identity \(\delta(f')=\sum \delta_{x_i}/|f''(x_i)|\) reduces the limiting object
  to a finite measure supported on critical points (amenable to formal proof assistants).
\end{itemize}

At the dynamics level (mechanics on a time line), the correct way to keep the logic honest is to time-slice:
at fixed discretization, the variational localization object is an ordinary finite-dimensional Dirac delta.
The open work is the controlled refinement limit under which these finite-dimensional statements produce a stable
continuum object.
In this repository, that refinement control is represented by the projective/cylinder framework and the scale-control
invariants (SD and \(\tau_\mu\)-type covariance).
\begin{itemize}[leftmargin=1.4em]
\item time-slicing makes \(\delta(\delta S/\delta q)\) exact as \(\delta(\nabla S_N)\) at fixed discretization,
\item continuum claims require limit control under refinement, tracked by projective/cylinder programs and scale-flow constraints.
\end{itemize}

\subsection{Field Level: \(d=2\) Closure, \(d=3\) Bridge, \(d=4\) Obstruction}
At the field level, Claim~1 becomes dimension-sensitive and must be treated as a dimension-gated existence program.
The current status by dimension is:

\paragraph{\(d=2\) closure.}
We proved a fully explicit interacting field-indexed closure in an ultralocal \(\phi^4\) class
(\path{research/workspace/notes/theorems/2026-02-09-claim1-d2-ultralocal-phi4-closure.md}).
Because the action is sitewise additive, cylinder expectations become exactly scale-independent once the cylinder sites
are distinct, and the continuum limit exists without renormalization. In that model we also obtain a field-level SD
identity by one-dimensional integration by parts and exact \(c\)-invariance along \(\tau_\mu\)-orbits.
\begin{itemize}[leftmargin=1.4em]
\item \(d=2\) ultralocal interacting \(\phi^4\) model: exact cylinder stabilization and continuum existence,
  field-level SD identity, and exact \(c\)-invariance along \(\tau_\mu\)-orbits.
\item interpretation: this is non-Gaussian but ultralocal, serving as a controlled baseline rather than full local QFT.
\end{itemize}

\paragraph{\(d=3\) bridge candidate.}
The next rung is \(d=3\) with local propagation reintroduced by a nearest-neighbor gradient term.
We stated a precise candidate theorem in \path{research/workspace/notes/theorems/2026-02-09-claim1-d3-intermediate-bridge-candidate.md}
and made the closure obligations explicit (B1--B5). The most important point is that the \(d=2\) proof does not even
begin to lift until one has moment/tightness and denominator control uniformly in \((a,L)\), and a quantitative
small-\(\kappa\) continuity bound for local cylinders.
\begin{itemize}[leftmargin=1.4em]
\item add nearest-neighbor gradient coupling \(\kappa\ge 0\) and seek a small-\(\kappa\) regime \([0,\kappa_*]\),
  with explicit proof obligations:
  B1 moments, B2 tightness, B3 non-vanishing denominator, B4 SD insertion control, B5 \(\kappa\)-continuity.
\item Lean work mechanizes B5-shaped increment bounds in abstract and in a finite exponential-family toy class;
  the field-level model bridge remains open.
\end{itemize}

\paragraph{\(d=4\) obstruction sheet.}
The obstruction sheet \path{research/workspace/notes/theorems/2026-02-09-claim1-d4-lift-obstruction-sheet.md}
records why naive lifts fail once local propagation and regulator removal are taken seriously:
loss of product structure, non-closed SD hierarchies, and renormalization dependence. In \(d=4\) one must add
explicit renormalization flow control and a nontriviality criterion; otherwise one can at best obtain a theorem-grade
scoped statement (regulated and/or cylinder-level) without a claim of a nontrivial interacting continuum theory.
\begin{itemize}[leftmargin=1.4em]
\item restoring local propagation destroys product structure and exact cylinder decoupling;
  SD hierarchy is no longer one-site closed.
\item regulator removal requires explicit renormalization flow control and nontriviality criteria.
\item half-density statements must be split into kinematic vs dynamical (AR note).
\end{itemize}

\section{How We Continue (Roadmap)}
\label{sec:roadmap}

\subsection{Immediate Next Targets}
\textbf{Immediate queue (scheme).}
\begin{itemize}[leftmargin=1.4em]
\item AN-18 (Lean): reduce finite exponential-family increment hypotheses by proving automatic regularity and non-vanishing conditions.
\item Field-level: close at least one genuine beyond-ultralocal \(d=3\) statement by addressing B1--B5 systematically.
\item Cross-level: keep half-density composition results separate from continuum existence, and ensure every ``lift'' names its missing analytic gates.
\end{itemize}

\subsection{Longer-Horizon Conjectures and Risks}
\textbf{Core missing analytic ingredients (scheme).}
\begin{itemize}[leftmargin=1.4em]
\item tightness/precompactness mechanisms for cylinder marginals in local interacting field models,
\item robust non-vanishing control for oscillatory partition normalizations on complex parameter domains,
\item SD pass-through with gradient terms (control of insertions and boundary terms),
\item \(d=4\) renormalization flow control and nontriviality criteria,
\item reconstruction to Minkowski (when pursuing that branch).
\end{itemize}

\section{Appendices}
\label{sec:appendix}

\subsection{Glossary}
\textbf{Core glossary (scheme).}
\begin{itemize}[leftmargin=1.4em]
\item \(c\)-parameter:
  \(c := (\eta - i/h)\kappa\).
\item \(c\)-invariant:
  invariant under parameter changes that keep \(c\) fixed; constant on \(\tau_\mu\)-orbits.
\item \(\tau_\mu\) flow:
  \(\tau_\mu:(\kappa,\eta,h)\mapsto(\mu\kappa,\eta/\mu,\mu h)\), \(\mu>0\).
\item de-regularization:
  one-sided limit \(\eta\to0^+\) from damped oscillatory weights to purely oscillatory ones.
\item SD identity (scoped finite form):
  \(\langle \partial_i\psi\rangle_c = c\,\langle \psi\,\partial_i S\rangle_c\).
\item half-density (here):
  amplitude-level object whose modulus-square yields density-level quantities; natural for groupoid convolution kernels.
\end{itemize}

\subsection{Reproducibility (Commands and Entry Points)}
\textbf{Lean build (scheme).}
\begin{verbatim}
cd research/workspace/proofs
/Users/arivero/.elan/bin/lake update
/Users/arivero/.elan/bin/lake build Claim1lean
\end{verbatim}

\textbf{Selected diagnostics (scheme).}
\begin{verbatim}
python3.12 research/workspace/simulations/claim1_d2_ultralocal_phi4_closure_check.py
python3.12 research/workspace/simulations/claim1_halfdensity_kinematic_dynamic_split_check.py
\end{verbatim}

\subsection{File Map}
\textbf{Curated index (scheme).}
\begin{itemize}[leftmargin=1.4em]
\item source: \path{conv\_patched.md}, \path{conv\_patched.pdf}.
\item audit: \path{research/workspace/notes/audits/2026-02-08-top10-claim-audit.md}.
\item compass: \path{research/workspace/notes/2026-02-09-core-goal-compass.md}.
\item Claim~1 reports: \path{research/workspace/reports/2026-02-08-claim1-variational-delta-note.tex},
  \path{research/workspace/reports/2026-02-09-claim1-scoped-complete-proof.tex}.
\item synthesis report: \path{research/workspace/reports/2026-02-09-newton-action-path-integral-lecture.md}.
\item Lean: \path{research/workspace/proofs/Claim1lean.lean} and \path{research/workspace/proofs/Claim1lean/}.
\item theorem notes index: \path{research/workspace/notes/README.md}.
\end{itemize}

\end{document}
