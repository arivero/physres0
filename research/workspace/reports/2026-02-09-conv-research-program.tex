\documentclass[11pt]{article}
\usepackage[a4paper,margin=1in]{geometry}
\usepackage{amsmath,amssymb,amsthm,mathtools}
\usepackage{bm}
\usepackage{enumitem}
\usepackage{hyperref}

\newtheorem{theorem}{Theorem}[section]
\newtheorem{lemma}[theorem]{Lemma}
\newtheorem{proposition}[theorem]{Proposition}
\newtheorem{corollary}[theorem]{Corollary}
\newtheorem{definition}[theorem]{Definition}
\newtheorem{remark}[theorem]{Remark}

\newcommand{\R}{\mathbb{R}}
\newcommand{\C}{\mathbb{C}}
\newcommand{\eps}{\varepsilon}
\newcommand{\e}{\mathrm{e}}
\newcommand{\Cyl}{\mathrm{Cyl}}
\newcommand{\Cov}{\mathrm{Cov}}
\newcommand{\path}[1]{\texttt{#1}}

\newcommand{\TODO}[1]{\par\noindent\textbf{TODO:} #1\par}

\title{From \path{conv\_patched.md} to a Research Program:\\
Foundational Goals, Proof Targets, Work Done, and Next Steps}
\author{}
\date{2026-02-09 (CET)}

\begin{document}
\maketitle

\begin{abstract}
\TODO{Write a 10--15 line abstract once the article is filled.}
This report is the canonical ``what we are doing'' document for the
\path{conv\_patched.md} conversation and the subsequent research carried out in this repository.
It explains:
(i) what was already known in the source conversation,
(ii) what we aimed to research,
(iii) what we expect to prove (and what counts as a proof here),
(iv) what has been done so far (theorem notes, Lean proofs, scripts, and reports),
and (v) how we intend to continue.
\end{abstract}

\tableofcontents

\section{Reading Instructions (for Humans and Future Agents)}
\label{sec:reading}

\begin{itemize}[leftmargin=1.4em]
\item Primary source: \path{conv\_patched.md} and its rendered companion \path{conv\_patched.pdf}.
\item Research compass: \path{research/workspace/notes/2026-02-09-core-goal-compass.md}.
\item Top-10 claim audit: \path{research/workspace/notes/audits/2026-02-08-top10-claim-audit.md}.
\item Claim 1 scoped proof report: \path{research/workspace/reports/2026-02-09-claim1-scoped-complete-proof.tex}.
\item Lean proof workspace: \path{research/workspace/proofs/} (build via \path{/Users/arivero/.elan/bin/lake}).
\end{itemize}

\TODO{Add a short ``how to use this report'' paragraph after the rest is written.}

\section{The Source Conversation: What Was Known and What Was Claimed}
\label{sec:conv}

\subsection{Conversation Inventory}
\TODO{Summarize the major themes in \path{conv\_patched.md}:}
\begin{itemize}[leftmargin=1.4em]
\item action reduction and geometry-of-force: SR/GR/gauge central-force taxonomy,
\item variational/distribution core: $\delta(\nabla S)$, ``delta of first variation'', and scaling channels,
\item half-density / ``halved amplitude'' viewpoint and tangent-groupoid composition intuition,
\item scale/refinement control: \(\tau_\mu\), Schwinger--Dyson identities, and de-regularization \(\eta\to0^+\).
\end{itemize}

\subsection{Canonical Anchors in \path{conv\_patched.md}}
\TODO{Insert the most important line references used in later sections.}

\section{Research Aims (North Star)}
\label{sec:aim}

\subsection{Goal 1 as a Three-Level Program}
\TODO{Explain the three levels, and that half-density is optional at each level:}
\begin{enumerate}[label=(\arabic*),leftmargin=1.8em]
\item statics (``0D action''): finite-dimensional oscillatory integrals and distributional localization,
\item dynamics (QM): action integrated over a time coordinate \(t\) and discrete-time limits,
\item fields (QFT): action integrated over spacetime coordinates with dimension-dependent existence gates.
\end{enumerate}

\subsection{Dimension-Gated Field Existence Ladder}
\TODO{Explain why \(d=2\) is tractable, \(d=3\) is a bridge target, and \(d=4\) is the frontier.}

\section{What We Expect to Prove (Proof Targets and Success Criteria)}
\label{sec:targets}

\subsection{Claim 1 (Foundational Claim)}
\TODO{State the core Claim 1 hypothesis/claim in repository language, with a clear notion of ``proved'' vs ``scoped proof'' vs ``heuristic''.}

\subsection{Auxiliary Claims (2--10)}
\TODO{State which non-Claim-1 results are now theorem-grade and why they matter to the narrative.}

\subsection{What Counts as a Proof Here}
\TODO{Explain the validation contract: assumptions, units, symmetry, and independent cross-check paths (Lean/symbolic/numeric).}

\section{What Has Been Done So Far (Artifacts and Results)}
\label{sec:done}

\subsection{Audit Results}
\TODO{Summarize the current top-10 audit table and the current ``next'' item.}

\subsection{Theorem Notes and Reports}
\TODO{Curate the most important \path{research/workspace/notes/theorems/*.md} and \path{research/workspace/reports/*} artifacts.}

\subsection{Lean Formalization (Machine-Checked Spine)}
\TODO{Summarize the Lean modules and the logical chain they support.}

\subsection{Simulation and Symbolic Checks}
\TODO{Summarize the Python scripts used as diagnostics and cross-checks.}

\section{Current Status of Claim 1 (Closed Pieces, Open Gaps)}
\label{sec:status}

\subsection{The Scoped Complete Proof and Its Scope}
\TODO{Explain what \path{2026-02-09-claim1-scoped-complete-proof.tex} proves and what it does not.}

\subsection{Statics and Dynamics: Where the Delta-of-Variation Logic Is Theorem-Grade}
\TODO{Summarize what is proved at static and discrete-time levels, and what is conjectural in the continuum.}

\subsection{Field Level: \(d=2\) Closure, \(d=3\) Bridge, \(d=4\) Obstruction}
\TODO{Summarize \path{claim1-d2-ultralocal-phi4-closure.md}, \path{claim1-d3-intermediate-bridge-candidate.md}, and \path{claim1-d4-lift-obstruction-sheet.md}.}

\section{How We Continue (Roadmap)}
\label{sec:roadmap}

\subsection{Immediate Next Targets}
\TODO{List the next research targets in the AN queue, and how they tie back to Goal 1.}

\subsection{Longer-Horizon Conjectures and Risks}
\TODO{List key missing ingredients (tightness, reconstruction, renormalization, dimension-dependent breakdowns).}

\section{Appendices}
\label{sec:appendix}

\subsection{Glossary}
\TODO{Copy/adapt the core glossary (\path{research/workspace/notes/2026-02-09-foundational-glossary.md}) into TeX form.}

\subsection{Reproducibility (Commands and Entry Points)}
\TODO{Write the minimal commands to build Lean and run key scripts.}

\subsection{File Map}
\TODO{Provide a curated index of the most important files produced so far.}

\end{document}
