\documentclass[11pt]{article}
\usepackage[a4paper,margin=1in]{geometry}
\usepackage{amsmath,amssymb,amsthm,mathtools}
\usepackage{hyperref}
\usepackage{upquote}

\newtheorem{theorem}{Theorem}
\newtheorem{proposition}{Proposition}
\newtheorem{remark}{Remark}

\title{Field-Theory Consistency Across Spacetime Dimension:\\
General-Dimension Program, Gates, and Literature Anchors}
\author{}
\date{2026-02-09}

\begin{document}
\maketitle

\begin{abstract}
This manuscript starts Goal 1C (fields) as a general-dimension paper track.
It does not claim full interacting closure in all dimensions.
Instead it fixes a dimension-indexed theorem program with explicit gates:
regulated existence, continuum existence, and reconstruction.
Current progress includes an explicit \(d=2\) interacting closure in an ultralocal class,
plus a \(d=3\) local-interaction scoped closure in a compact-spin Euclidean subclass.
AN-24 removes the hard cutoff in that same branch inside a local-renormalized channel.
AN-26 plus AN-26B close SD test-side \(C_b^1\) extension with explicit
tail-control and insertion-moment verification in-branch.
AN-27 then transfers the widened local class to the oscillatory/de-regularized
branch under explicit non-vanishing and contour-envelope hypotheses.
AN-28 extends that oscillatory branch to disconnected nonlocal cylinders, and
AN-29 adds explicit refinement-Cauchy rates plus denominator bookkeeping for
their continuum extraction in the same scoped lane.
AN-30 extends this further to finite graph-indexed multi-block families with
explicit combinatorial constants and projective-consistency closure.
\end{abstract}

\section{Scope}

\subsection*{In scope}
\begin{enumerate}
\item one paper-level framework that treats \(d=2\), \(d=3\), \(d=4\), and \(d>4\) in a unified way,
\item explicit assumptions and failure gates by dimension,
\item source-backed baseline claims from standard literature.
\end{enumerate}

\subsection*{Out of scope}
\begin{enumerate}
\item claiming a fully closed interacting \(d=4\) continuum theorem in this pass,
\item replacing constructive estimates with formal kernel language alone.
\end{enumerate}

\section{Field Setup}

For a local scalar prototype on \(\mathbb{R}^d\),
\[
S_d[\Phi]=\int_{\mathbb{R}^d}\left[\frac12 |\nabla\Phi|^2+\frac{m^2}{2}\Phi^2+\frac{\lambda}{4!}\Phi^4\right]\,d^dx.
\]
At finite cutoff/volume level (symbolically \(\Lambda,a,L\)),
\[
\omega_{c,\Lambda,a,L}(F)=
\frac{\int e^{-cS_{d,\Lambda,a,L}[\Phi]}F[\Phi]\,D\Phi}
{\int e^{-cS_{d,\Lambda,a,L}[\Phi]}\,D\Phi},
\qquad \Re c>0.
\]

\section{Three Gates (Mandatory)}

For any dimension branch, the paper only upgrades to ``closed'' when all three gates are met:
\begin{enumerate}
\item \textbf{G1 Regulated existence}: finite cutoff and finite volume channel is well-defined.
\item \textbf{G2 Continuum existence}: limits in cutoff/volume exist with nontrivial content.
\item \textbf{G3 Reconstruction}: Euclidean channel maps to the intended physical QFT object class.
\end{enumerate}

\section{Dimension-Indexed Program Claim}

\begin{proposition}[General-dimension program schema]
Let \(d\ge2\). For each dimension branch \(\mathcal{B}_d\), define closure status
\[
\mathrm{Closed}(\mathcal{B}_d)\iff G1(\mathcal{B}_d)\wedge G2(\mathcal{B}_d)\wedge G3(\mathcal{B}_d).
\]
Then:
\begin{enumerate}
\item \(d=2\) is the first full-closure candidate branch.
\item \(d=3\) is the next branch with superrenormalizable control targets.
\item \(d=4\) remains a frontier branch where theorem-grade scoped statements and explicit open assumptions must be separated.
\item \(d>4\) is treated primarily as EFT/mean-field/triviality-guided branch unless a UV-complete model class is fixed.
\end{enumerate}
\end{proposition}

\begin{remark}
This proposition is a program statement, not a universal theorem of existence.
Its role is to prevent overclaiming and to force assumption-explicit upgrades.
\end{remark}

\section{Current Status by Dimension}

\begin{enumerate}
\item \textbf{\(d=2\):} scoped interacting closure is already available in an ultralocal
\(\phi^4\) class (existence of the cylinder limit state, SD pass-through, and exact
\(c\)-invariance in that class).
\item \textbf{\(d=3\):} nearest-neighbor local interactions now have scoped closure in a
compact-spin Euclidean subclass (B1-B4 + renormalized B5 input closed there), while
hard-cutoff removal is now closed in a local-renormalized channel and class-widening
is closed in this scoped Euclidean branch through AN-25/AN-26/AN-26B.
AN-27 transfers this widened class to the oscillatory/de-regularized branch;
AN-28/AN-29 then extend to disconnected nonlocal cylinders with explicit
refinement-Cauchy and denominator-rate control in that same scoped branch.
AN-30 upgrades this to finite graph-indexed multi-block families with explicit
combinatorial rates and projective consistency in the refinement limit.
\item \textbf{\(d=4\):} frontier branch; no full interacting closure claim is made here.
\item \textbf{\(d>4\):} treated as EFT/triviality-guided unless a UV-complete model is fixed.
\end{enumerate}

\section{What the Current Lean Chain Supports}

Machine-checked finite-model modules provide reusable inequality templates for
small-parameter increment control and regularity bookkeeping.

Current limitation:
these modules do not by themselves discharge field-level G2/G3.
They feed the field program only when translated into dimension-indexed bound propositions.

\section{Current \(d=3\) Interacting Branch (Scoped Closure + Open Gap)}

Work in the local periodic finite-volume \(\phi^4\) lattice class with nearest-neighbor coupling,
Euclidean \(c\in[c_0,c_1]\subset(0,\infty)\), bounded local cylinders
\(F(\phi)=f(\phi|_B)\), and local edge insertion
\[
G_B=\sum_{\langle x,y\rangle\in E_B}(\phi_x-\phi_y)^2.
\]

A first finite-volume estimate layer gives explicit constants
\[
K_F=2\|F\|_\infty,\qquad
M_{B,a}=\frac{4|E_B|}{c_0m_0^2a^3},
\]
with
\[
|F-\omega(F)|\le K_F,\qquad \omega(G_B)\le M_{B,a},
\]
uniform in \(L\) and \(\kappa\in[0,\kappa_*]\).

A renormalized insertion channel removes explicit \(a^{-3}\) growth from this B5b input:
\[
G_{B,a}^{\mathrm{ren}}:=a^3G_B,\qquad
\omega(G_{B,a}^{\mathrm{ren}})\le M_B^{\mathrm{ren}},
\qquad
M_B^{\mathrm{ren}}=\frac{4|E_B|}{c_0m_0^2},
\]
uniform in \(a,L,\kappa\).

The AN-23 compact-spin branch discharges these four obligations in one explicit
interacting Euclidean subclass:
\begin{itemize}
\item \textbf{B1} uniform local moment bounds in \((a,L)\),
\item \textbf{B2} local tightness/precompactness of cylinder marginals,
\item \textbf{B3} denominator non-vanishing on the working \(c\)-domain,
\item \textbf{B4} SD insertion-control pass-through to the continuum limit.
\end{itemize}

Thus the AN-22 candidate is upgraded to scoped closure in that subclass.
AN-24 then removes the hard compact-spin cutoff \(R\to\infty\) while keeping
B1-B4 in the local-renormalized compact-support channel.
AN-25 closes observable-side widening \(C_c\to C_b\).
AN-26 + AN-26B close SD test-side widening \(C_c^1\to C_b^1\) by combining the
Holder/Markov tail criterion with an explicit \(q=4/3\) insertion-moment bound.

\section{Literature Anchors for the Dimension Ladder}

\subsection*{Reconstruction and Euclidean axioms}
\begin{enumerate}
\item Osterwalder--Schrader I (1973), Euclidean axioms:
\href{https://doi.org/10.1007/BF01645738}{doi:10.1007/BF01645738}.
\item Osterwalder--Schrader II (1975), reconstruction continuation:
\href{https://doi.org/10.1007/BF01608978}{doi:10.1007/BF01608978}.
\end{enumerate}

\subsection*{Scale/refinement framework}
\begin{enumerate}
\item Wilson--Kogut (1974), renormalization-group framing:
\href{https://doi.org/10.1016/0370-1573(74)90023-4}{doi:10.1016/0370-1573(74)90023-4}.
\end{enumerate}

\subsection*{Constructive \(d=2\) baseline}
\begin{enumerate}
\item Guerra--Rosen--Simon (1975), \(P(\phi)_2\) Euclidean construction:
\href{https://doi.org/10.2307/1970985}{doi:10.2307/1970985}.
\end{enumerate}

\subsection*{\(d=4\) frontier and triviality anchors}
\begin{enumerate}
\item Aizenman (1981), \(\phi^4_d\) triviality/mean-field features for \(d>4\):
\href{https://doi.org/10.1103/PhysRevLett.47.1}{doi:10.1103/PhysRevLett.47.1}.
\item Aizenman--Duminil-Copin (2021), marginal triviality in critical \(4d\) scaling limits:
\href{https://doi.org/10.4007/annals.2021.194.1.3}{doi:10.4007/annals.2021.194.1.3}.
\end{enumerate}

\section{Immediate Next Scientific Step}

The next theorem target is:
\begin{enumerate}
\item extend AN-30 from finite graph-indexed families to uniformly locally
finite exhaustion families (AN-31),
\item keep summability-weighted combinatorial constants explicit while preserving
projective-consistency bookkeeping across exhaustion levels.
\end{enumerate}

\section{Validation Contract}

\subsection*{Assumptions}
\begin{enumerate}
\item model class and dimension branch are explicit in each statement,
\item closure status is always reported against G1/G2/G3, never implied.
\end{enumerate}

\subsection*{Units and dimensions}
\begin{enumerate}
\item \(S_d\) has action dimension,
\item phase/weight argument is dimensionless.
\end{enumerate}

\subsection*{Independent checks}
\begin{enumerate}
\item theorem/estimate channel (Lean where feasible, analytic where not),
\item bibliography check against standard references above,
\item executable finite surrogate checks:
\texttt{python3.12 research/workspace/simulations/claim1\_d3\_finite\_volume\_centered\_moment\_bound\_check.py}.
\item renormalized-channel check:
\texttt{python3.12 research/workspace/simulations/claim1\_d3\_renormalized\_moment\_channel\_check.py}.
\item continuum-branch proxy check:
\texttt{python3.12 research/workspace/simulations/claim1\_d3\_an22\_continuum\_branch\_proxy\_check.py}.
\item AN-23 compact-spin closure diagnostics:
\texttt{python3.12 research/workspace/simulations/claim1\_d3\_an23\_compact\_spin\_closure\_check.py}.
\item AN-24 hard-cutoff lift diagnostics:
\texttt{python3.12 research/workspace/simulations/claim1\_d3\_an24\_cutoff\_lift\_check.py}.
\item AN-25 local class-extension diagnostics:
\texttt{python3.12 research/workspace/simulations/claim1\_d3\_an25\_class\_extension\_check.py}.
\item AN-26 SD-test tail insertion-control diagnostics:
\texttt{python3.12 research/workspace/simulations/claim1\_d3\_an26\_tail\_insertion\_control\_check.py}.
\item AN-26B insertion \(L^{4/3}\)-moment diagnostics:
\texttt{python3.12 research/workspace/simulations/claim1\_d3\_an26b\_insertion\_lq\_moment\_check.py}.
\item AN-27 oscillatory/de-regularized transfer diagnostics:
\texttt{python3.12 research/workspace/simulations/claim1\_d3\_an27\_oscillatory\_dereg\_transfer\_check.py}.
\item AN-28 two-block nonlocal transfer diagnostics:
\texttt{python3.12 research/workspace/simulations/claim1\_d3\_an28\_nonlocal\_cylinder\_transfer\_check.py}.
\item AN-29 nonlocal refinement-Cauchy diagnostics:
\texttt{python3.12 research/workspace/simulations/claim1\_d3\_an29\_nonlocal\_continuum\_cauchy\_check.py}.
\item AN-30 multiblock projective-consistency diagnostics:
\texttt{python3.12 research/workspace/simulations/claim1\_d3\_an30\_multiblock\_projective\_consistency\_check.py}.
\end{enumerate}

\subsection*{Confidence statement}
This document is a theorem-program and source-anchored roadmap.
It now includes scoped \(d=3\) closure with hard-cutoff lift and
AN-25/AN-26/AN-26B class-extension closure in a scoped Euclidean branch.
AN-27 closes oscillatory/de-regularized transfer; AN-28/AN-29 extend this to
disconnected nonlocal cylinders with explicit refinement-Cauchy bookkeeping in
the same scoped branch; AN-30 extends further to finite graph-indexed
multi-block projective consistency with explicit combinatorial constants.
Full global continuum interacting closure remains open.

\section{Reproducibility Metadata}

\begin{itemize}
\item date anchor: 2026-02-09 (US),
\item build toolchain: \texttt{/Library/TeX/texbin/pdflatex} (TeX Live 2025),
\item safe build script:
\texttt{\textasciitilde/.codex/skills/pdflatex-safe-build/scripts/build\_pdflatex\_safe.sh}.
\end{itemize}

\end{document}
